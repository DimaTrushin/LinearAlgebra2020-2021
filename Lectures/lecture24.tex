\ProvidesFile{lecture24.tex}[Лекция 24]


\subsection{Двойственность для линейных отображений и операторов}

В утверждении ниже, мы предполагаем, что все ортогональные дополнения берутся относительно естественной билинейной формы, то есть $\langle, \rangle\colon V^*\times V\to F$, где $(\xi, v)\mapsto \langle \xi, v\rangle := \xi(v)$.
В данном случае, я не буду различать левые и правые ортогональные дополнения, так как понятно, в каком пространстве находятся соответствующие подпространства.
В частности, если $W\subseteq V$, то $W^\bot\subseteq V^*$ и наоборот, если $W\subseteq V^*$, то $W^\bot \subseteq V$.

Давайте сделаю еще одно полезное замечание относительно изоморфизма $\phi\colon V\to V^{**}$.
Посмотрим на следующую диаграмму
\[
\xymatrix@R=5pt{
	{V^*\times V}\ar[dd]_{\Identity\times \phi}\ar[rd]&{}\\
	{}&{F}\\
	{V^*\times V^{**}}\ar[ru]&{}\\
}
\quad
\xymatrix@R=5pt{
	{(\xi, v)}\ar@{|->}[dd]\ar@{|->}[rrd]&{}&{}\\
	{}&{\phi_v(\xi)}\ar@{=}[r]&{\xi(v)}\\
	{(\xi,\phi_v)}\ar@{|->}[ru]&{}&{}\\
}
\]
Здесь горизонтальные стрелки -- это естественные билинейные формы применения функции к вектору.
Справа показано, как стрелки действуют на элементах.
И равенство показывает, что диаграмма коммутативна.
А это означает, что если мы возьмем подпространство $W\subseteq V^*$, то мы можем проделать с ним две процедуры: 
\begin{enumerate}
\item В начале возьмем ортогональное дополнение относительно верхней формы и получим $W^\bot \subseteq V$, а потом применим $\phi$ и получим $\phi(W^\bot)\subseteq V^{**}$.

\item Сразу возьмем ортогональное дополнение относительно нижней формы и получим $W^\bot \subseteq V^{**}$.
\end{enumerate}
Так вот, коммутативность диаграммы выше означает, что результаты этих двух операций совпадают, то есть полученные подпространства в $V^{**}$ будут одинаковыми.
Я утверждаю, что это объясняется методом пристального взгляда,%
\footnote{Но как обычно слепым для прозрения рекомендую воспользоваться бумажкой и ручкой.
Пригодится в любом случае: либо написать доказательство, либо вытереть слезы отчаяния.}
и мы воспользуемся этим наблюдением в доказательстве следующего результата.

\begin{claim}[Альтернатива Фредгольма]
\label{claim::Fredholm}
Пусть $\varphi\colon V\to U$ -- некоторое линейное отображение и $\varphi^*\colon U^*\to V^*$ -- сопряженное к нему.
Тогда
\begin{enumerate}
\item $(\Im \varphi)^\bot = \ker \varphi^*$.

\item $(\ker \varphi)^\bot = \Im \varphi^*$.
\end{enumerate}
\end{claim}
\begin{proof}
(1) Следующая цепочка равенств проводит доказательство:
\[
\ker\varphi^* =\{\xi \in U^*\mid \varphi^*(\xi) = 0\} = \{\xi \in U^*\mid \xi \varphi = 0\} = \{\xi\in U^*\mid \xi(\Im \varphi) = 0\} = \{\xi\in U^*\mid \langle \xi, \Im\varphi\rangle = 0\} = (\Im \varphi)^\bot
\]


Теперь докажем, что (1) влечет (2).
По утверждению~\ref{claim::DualitySpaces} пункты~(2) и~(3) равенство $(\ker \varphi)^\bot = \Im \varphi^*$ в пространстве $V^*$  эквивалентно равенству $\ker \varphi = (\Im \varphi^*)^\bot$ в пространстве $V$.
Теперь применим к этому равенству изоморфизм $\phi\colon V\to V^{**}$ (утверждение~\ref{claim::CanonicalIsomorphism}).
Тогда $\ker \varphi$ перейдет в $\ker \varphi^{**}$ (это сразу следует из утверждения~\ref{claim::CanonicalIsomorphism}).
Кроме того, $(\Im \varphi^*)^\bot \subseteq V$, где ортогональное дополнение берется относительно формы $V^*\times V\to F$, перейдет в $(\Im \varphi^*)^\bot\subseteq V^{**}$, где ортогональное дополнение берется относительно формы $V^*\times V^{**}\to F$.
Это получается из замечания перед формулировкой утверждения выше.
Потому нам достаточно показать, что $(\Im \varphi^*)^\bot = \ker(\varphi^*)^*$ в пространстве $V^{**}$.
А это равносильно (1), примененному к отображению $\varphi^*\colon U^*\to V^*$.
\end{proof}

\paragraph{Замечания}
\begin{itemize}
\item Я хочу добавить пару замечаний о пользе утверждения выше.
Не надо ожидать, что оно даст вам супер тайное знание, которое решит все проблемы.
Нет, это скорее доказательство бессмысленного очевидного факта.
Чтобы понять, почему он бессмысленный выберете базис в $V$, потом двойственный к нему в $V^*$ и переформулируйте все на матричном языке.
Очень удивитесь тому, какую элементарную дичь мы тут с вами делаем.
А раз так, то надо бы объяснить, какого я вас тут мучаю заведомо более дурацким доказательством и совсем бессмысленным рассуждением.

Так вот, самая большая польза от этого утверждения -- сломать мозг.
А именно, я хочу продемонстрировать вам конструкции в рамках линейной алгебры, которые мозгу тяжело воспринимать, про которые сложно думать, но которые часто встречаются в математике.
Причина, почему вам сложно, проста -- у вас нет интуиции про эти объекты и вы никогда не проделывали подобные рассуждения и не видели таких конструкций.
И потому самое полезное из произошедшего -- вы все это увидели и у вас был шанс в этом разобраться или попробовать разобраться.
Вот именно необходимость разобраться в абстрактной чуши и есть цель этого утверждения, как и его доказательства.

\item Чуть выше я уже рекомендовал все перевести на матричный язык.
Другое хорошее упражнение -- попробовать расписать явно условие $(\ker \varphi)^\bot = \Im \varphi^*$ и доказать его в лоб без применения двойственности.
Это тоже очень полезное упражнение.
Вообще, очень полезно видеть разные способы доказать одно и то же, это помогает лучше понять место знания в экосистеме изученного.
\end{itemize}

В случае, когда $\varphi\colon V\to V$ является оператором, то у него намного больше характеристик, чем просто ядро и образ.
Следующее утверждение подытоживает все знания об общих характеристиках воедино.

\begin{claim}
Пусть $\varphi\colon V\to V$ -- некоторый линейный оператор над полем $F$.
Тогда
\begin{enumerate}
\item $\tr \varphi = \tr \varphi^*$.

\item $\det \varphi = \det \varphi^*$.

\item $\chi_\varphi = \chi_{\varphi^*}$.

\item $f_{\text{min},\,\varphi} = f_{\text{min},\,\varphi^*}$.

\item $\spec_F(\varphi) = \spec_F(\varphi^*)$.

\item $\spec_F^I(\varphi) = \spec_F^I(\varphi^*)$.

\item если $U\subseteq V$ -- $\varphi$-инвариантное, то $U^\bot\subseteq V^*$ -- $\varphi^*$-инвариантное.
\end{enumerate}
\end{claim}
\begin{proof}
Все утверждения кроме последнего следуют из утверждения~\ref{claim::DualHomMatrix}, которое гласит, что при правильном выборе базисов в $V$ и $V^*$ матрицы $\varphi$ и $\varphi^*$ отличаются транспонированием.

(7) Нам дано $\varphi(U)\subseteq U$, а надо показать, что $\varphi^*(U^\bot)\subseteq U^\bot$.
Пусть $\xi\in U^\bot$, то есть $\xi(U) = 0$, надо проверить, что $\varphi^*(\xi)(U) = 0$.
То есть надо проверить, что $(\xi \varphi)(U) = 0$.
Но $(\xi \varphi) (U) = \xi(\varphi(U))\subseteq \xi(U) = 0$, что и требовалось.
\end{proof}

\paragraph{Замечания}
\begin{itemize}
\item Таким образом изучение оператора $\varphi$ это тоже самое, что изучение оператора $\varphi^*$, если при этом <<перевернуть>> все пространства.
То есть любой вопрос про $\varphi^*$ можно переформулировать в терминах $\varphi$, заменив размерности подпространств в формулировках на $n$ минус размерность (здесь $n$ -- размерность объемлющего пространства $V$).

\item На самом деле для $\varphi$ и $\varphi^*$ совпадают жордановы нормальные формы.
Действительно, если вы транспонируете матрицу в жордановой нормальной форме, то результирующая матрица будет иметь ту же самую жорданову нормальную форму.

\item Хорошее упражнение для ума: пусть $e_1,\ldots,e_n$ -- жорданов базис для $\varphi$, а $e^1,\ldots,e^n$ -- двойственный к нему базис.
Как из $e^1,\ldots,e^n$ получить жорданов базис для $\varphi^*$?
\end{itemize}

\paragraph{Примеры}
\begin{enumerate}
\item Пусть $\varphi\colon V\to V$ такой, что характеристический многочлен $\varphi$ раскладывается на линейные множители.
А это означает, что $V$ раскладывается в прямую сумму корневых, то есть $V = V^{\lambda_1}\oplus \ldots \oplus V^{\lambda_r}$ (утверждение~\ref{claim::RootSpaceDec}).
Так как характеристический многочлен у $\varphi^*$ такой же, то $V^* = (V^*)^{\lambda_1}\oplus \ldots \oplus (V^*)^{\lambda_r}$.
Давайте покажем, что $(V^{\lambda_1})^\bot = (V^*)^{\lambda_2}\oplus \ldots \oplus (V^*)^{\lambda_r}$.

Самый простой способ сделать это такой.
Рассмотрим многочлен $p(t) = (t-\lambda_1)^{k_1}$, где $k_1$ -- кратность $\lambda_1$ в $\chi_\varphi$.
Тогда $V^{\lambda_1} = \ker p(\varphi)$ (это, например, следует из утверждения~\ref{claim::RootMultGeom}, однако, можно просто взять достаточно большое $k_1$ и можно обойтись леммой о стабилизации -- утверждение~\ref{claim::StabilityLemma}).
По утверждению~\ref{claim::Fredholm} о двойственности для линейных отображений, мы получаем, что $(\ker p(\varphi))^\bot = \Im (p(\varphi))^* = \Im p(\varphi^*)$.
Последнее равенство проверяется в лоб:
\[
p(\varphi)^* =  \Bigl(\sum_ka_k\varphi^k\Bigr)^* = \sum_ka_k \left(\varphi^k\right)^* = \sum_k a_k \left(\varphi^*\right)^k = p(\varphi^*)
\]
По утверждению~\ref{claim::IdealRootDec} для оператора $\varphi^*$ последний образ как раз и равен сумме оставшихся корневых.%
\footnote{Если не понятно, на что я тут ссылаюсь, то загляните в доказательство утверждения~\ref{claim::GenRootDec} пункт~(2), это должно снять все вопросы.}

\item Пусть $\varphi\colon V\to V$ -- линейный оператор над полем $\mathbb C$.
Тогда мы знаем, что обязательно найдется ненулевой собственный вектор $v\in V_\lambda$ для некоторого $\lambda\in \mathbb C$.
Последнее равносильно тому, что найдется инвариантное одномерное подпространство $\langle v \rangle$.
Давайте покажем, как с помощью двойственности автоматически найти инвариантное $n - 1$ мерное подпространство, где $n = \dim V$.
Пусть $U\subseteq V^*$ -- инвариантное одномерное для $\varphi^*$.
Такое существует, потому что у $\varphi^*$ есть собственный вектор (мы над полем $\mathbb C$).
В этом случае $U^\bot\subseteq V$ будет $n-1$ мерным и инвариантным для $\varphi$ по пункту~(7) предыдущего утверждения.
\end{enumerate}

\subsection{Классификационная задача для $\Bil(V,U)$}

\begin{claim}
Пусть $V$ и $U$ -- векторные пространства над полем $F$ размерностей $n$ и $m$, соответственно, и пусть $A,B\in \operatorname{M}_{n\,m}(F)$ -- произвольные матрицы.
Тогда эквивалентны следующие утверждения:
\begin{enumerate}
\item Матрицы $A$ и $B$ задают одну и ту же билинейную форму на $V$ и $U$ в разных базисах, т.е. существует билинейная форма $\beta\colon V\times U\to F$, два базиса в $V$: $e=(e_1,\ldots,e_n)$ и $e'=(e_1',\ldots,e_n')$ и два базиса в $U$: $f=(f_1,\ldots,f_m)$ и $f'=(f_1',\ldots,f_m')$ такие, что $A$ является матрицей $\beta$ в базисах $e$ и $f$, а $B$ является матрицей $\beta$ в базисах $e'$ и $f'$.

\item $\rk A = \rk B$.
\end{enumerate}
\end{claim}
\begin{proof}
Из (1) в (2) мы уже знаем, это корректность ранга билинейной формы.

Из (2) в (1).
Так как ранги матриц $A$ и $B$ одинаковые, мы можем найти такие обратимые матрицы $C,R \in \operatorname{M}_{n}(F)$ и $D, P\in \operatorname{M}_{m}(F)$, что
\[
A = C
\begin{pmatrix}
{E}&{0}\\
{0}&{0}
\end{pmatrix}
D,
\quad
B = R
\begin{pmatrix}
{E}&{0}\\
{0}&{0}
\end{pmatrix}
P
\]
А значит $B = RC^{-1}AD^{-1}P$, то есть $B = S^t A T$, где $S = (RC^{-1})^t$ и $T = D^{-1}P$.
Теперь нам надо найти билинейную форму  и две пары базисов.
Возьмем два произвольных базиса $e=(e_1,\ldots,e_n)$ в $V$ и $f=(f_1,\ldots,f_m)$ в $U$ и положим $\beta\colon V\times U \to F$ по правилу $\beta(v, u) = x^t A y$, где $v = ex$ и $x\in F^n$, $u = f y$ и $y\in F^m$.
После положим $e' = e S$ и $f' = fT$.
Так как $S$ и $T$ невырожденные матрицы, то $e'$ будет базисом $V$, а $f'$ -- базисом $U$.
Тогда в новой паре базисов матрица нашей билинейной формы будет $S^t A T$, что совпадает с $B$ по построению.
\end{proof}

Таким образом, как и в случае линейного отображения, две матрицы задают одну и ту же билинейную форму тогда и только тогда, когда их ранги совпадают.


\subsection{Структура векторного пространства на $\Bil(V,U)$}

До сих пор мы смотрели на $\Bil(V, U)$ как на множество билинейных форм, такой мешок, в котором аморфно лежат все наши замечательные симпатичные формочки, одна лучше другой.
На самом деле $\Bil(V, U)$ само является векторным пространством.
По-простому, это означает, что на $\Bil(V, U)$ есть хорошие операции сложения и умножения на константу.


\begin{definition}
Пусть $\beta_1,\beta_2\colon V\times U \to F$ -- две билинейные формы.
Мы хотим определить форму $\beta_1 + \beta_2$.
Это значит, что нам надо определить отображение $(\beta_1 + \beta_2)\colon V\times U \to F$.
Определим его по правилу
\[
(\beta_1 + \beta_2)(v, u) := \beta_1(v,u) + \beta_2(v,u)
\]
Если $\beta\colon V\times U \to F$ -- билинейная форма и $\lambda\in F$, тогда форму $(\lambda \beta)\colon V\times U \to F$ определим по правилу
\[
(\lambda\beta)(v, u) := \lambda \beta(v, u)
\]
\end{definition}

\paragraph{Замечания}
\begin{itemize}
\item В качестве упражнения я предлагаю проверить, что $\beta_1 + \beta_2$ и $\lambda\beta$ -- это не просто отображения из $V\times U$ в $F$, а билинейные формы, то есть удовлетворяют условиям определения~\ref{def::BilinearForms}.
Это объясняет, что определенные выше операции корректны, то есть их результат -- это тоже билинейная форма.

\item В качестве другого упражнения я предлагаю проверить, что множество $\Bil(V, U)$ является векторным пространством, то есть надо проверить аксиомы из определения~\ref{def::VectorSpace}.

\item Чтобы немного вывернуть мозги наизнанку, давайте вспомним, что билинейные формы имеют операторную запись, то есть вместо $\beta\colon V\times U\to F$ мы будем писать $\cdot_\beta\colon V\times U \to F$, при это $\beta(v, u) = v \cdot_\beta u$.
То есть билинейные формы -- это операции умножения векторов, которые в результате возвращают число.
Так вот, мы только что определили как складывать и умножать на числа операции умножения так, что они остаются операциями умножения!
Определения выше можно переписать так
\begin{align*}
v (\cdot_{\beta_1} + \cdot_{\beta_2}) u &= v \cdot_{\beta_1} u + v \cdot_{\beta_2} u\\
v(\lambda \cdot_\beta) u &= \lambda (v \cdot_\beta u)
\end{align*}
Неправда ли выносит мозг?
Это лишний раз говорит о том, что в математике очень важно правильно думать об объекте.
Сложение билинейных форм не представляется нам чем-то особенным, вот умение складывать операции умножения кажется диким.
Психологические барьеры надо уметь перепрыгивать.
\end{itemize}

Теперь вспомним, что фиксировав базисы в пространствах $V$ и $U$, каждая билинейная форма превращается в матрицу (утверждение~\ref{claim::BilinearMatrices}).
Так вот, заметим, что это превращение является измомрфизмом между векторными пространствами.
Для полноты картины я все же сформулирую сам результат.

\begin{claim}
%\label{claim::BilinearIsomMatrices}
Пусть $\beta\colon V\times U\to F$ -- некоторая билинейная форма,  $e = (e_1,\ldots,e_n)$ -- базис пространства $V$ и $f=(f_1,\ldots,f_m)$ -- базис пространства $U$.
Тогда отображение $\operatorname{Bil}(V,U)\to \operatorname{M}_{n\,m}(F)$ по правилу $\beta\mapsto B_\beta$ является изоморфизмом.
\end{claim}

Следующая наша задача разобраться со случаем аналогичным случаю оператора, а именно: билинейная форма определена на одном пространстве.


\newpage
\section{Билинейные формы на одном пространстве $\Bil(V)$}

В случае, когда линейное отображение определено на одном пространстве (линейный оператор), у нас в запасе намного больше конструкций и характеристик, чем в случае общего линейного отображения.
Аналогичная ситуация обстоит и с билинейными формами.
В случае, когда билинейная форма живет на одном пространстве у нас на много больше характеристик и поведение ее изучать несколько сложнее.
Ниже я буду рассказывать о билинейных формах на одном пространстве.
Окажется, что от части ситуация с билинейными формами технически сильно проще, чем случай линейных операторов.

\subsection{Симметричность и кососимметричность}

Первое отличие билинейных форм на одном пространстве от общего случая заключается в том, что мы можем подставлять один и тот же вектор как в качестве левого, так и в качестве правого аргумента.
Благодаря можно выделить классы симметричных и кососимметричных форм.

\begin{definition}
Пусть $\beta\colon V\times V\to F$ -- билинейная форма определенная на одном пространстве.
Тогда
\begin{itemize}
\item Будем говорить, что $\beta$ симметричная, если $\beta(u,v) = \beta(v,u)$ для любых $u,v\in V$.

\item Будем говорить, что $\beta$ кососимметрична, если $\beta(v,v) = 0$ для любого $v\in V$.
\end{itemize}
\end{definition}

\paragraph{Замечание}
Давайте обсудим свойство кососимметричности.
Вас должно было удивить такое дурацкое условие в определении.
Однако, теперь, когда мы уже взрослые и знаем разные поля, но толком про них ничего еще и не знаем, нам надо быть чуточку аккуратными.
Если $\beta(v,v) = 0$ для любого $v\in V$, тогда $\beta(v+u,v+u) = 0$ для любых $u,v\in V$.
Это значит
\[
0=\beta(v+u,v+u) = \beta(v,v) + \beta(v,u) + \beta(u,v)+\beta(u,u) = \beta(v,u) + \beta(u,v)
\]
То есть из этого условия вытекает, что $\beta(v,u) = -\beta(u,v)$.
Однако, если нам дано, что $\beta(v,u) = -\beta(u,v)$, то подставив $u = v$, мы получим $\beta(v,v) = -\beta(v,v)$, а значит $2\beta(v,v) = 0$.
И вот тут начинаются чудеса.
Бывают поля, в которых $2 = 0$.
В таких полях условие $\beta(v,v) = 0$ сильнее условия $\beta(v,u) = -\beta(u,v)$.
Если же $2$ обратимо в поле $F$, то оба условия равносильны.
Однако, в общем случае, правильное определение -- потребовать более сильное условие, которое и содержится в определении.


\subsection{Матрица билинейной формы}

Если $\beta\colon V\times V\to F$ -- билинейная форма на пространстве $V$ над полем $F$, то для определения ее матрицы нам достаточно выбрать только один базис.%
\footnote{Вообще говоря никто не мешает выбрать для одного и того же пространства два разных базиса для левого и правого аргумента.
И бывает, что так делать нужно.
Но вообще говоря не стоит.
В разных базисах один и те же векторы имеют разные координаты и у нас ломается главное, что мы имеем -- возможность сравнить левый и правый вектор методом взгляда.}
Если $e_1,\ldots,e_n$ -- базис пространства $V$, то матрица билинейной формы в этом базисе будет $B = (\beta(e_i, e_j))$.
Если мы переходим к новому базису $(e_1',\ldots,e_n') = (e_1,\ldots,e_n) C$, где $C$ -- обратимая матрица, и $B'$ -- матрица билинейной формы в новом базисе, то $B' = C^t B C$.

Обратите внимание, что в случае одного пространства все определяется одним базисом и в этом случае переход от одного базиса к другому на матричном языке идет по формуле $B' = C^t B C$.
Если внимательно посмотреть, как действуют матрицы элементарных преобразований (раздел~\ref{section::ElemMat}), то мы увидим, что переход от $B$ к $C^tBC$ означает совершить одинаковые элементарные преобразования над строками и столбцами матрицы $B$.
А это значит, что в отличие от операторов, нам не нужна никакая сложная теория с собственными значениями.
Достаточно действовать обычным Гауссом.
Чуть ниже мы увидим, что любую симметричную билинейную форму всегда можно диагонализовать.
Но вот сравнение диагональных форм будет требовать некоторых знаний про базовое поле и эту задачу мы решим лишь для полей $\mathbb R$ и $\mathbb C$.
Уже для поля $\mathbb Q$ и билинейных форм на двумерных пространствах мы получим нетривиальную задачу из теории чисел.

\subsection{Матричные характеристики билинейной формы}
\label{subsection::BilChar}


В случае линейного оператора мы поступали так: выбирали базисы и задавали их матрицами.
Потом считали какие-то характеристики этих самых матриц и показывали, что они не зависят от базисов.
Мы уже попробовали такой подход в случае общих билинейных форм (на двух разных пространствах).
Теперь давайте попробуем такой же подход в случае с билинейными формами

\paragraph{Ранг}

Пусть $\beta\colon V\times V\to F$ -- билинейная форма, которая в двух базисах имеет матрицы $B$ и $B'$, соответственно.
Тогда $B' = C^t B C$ для некоторой невырожденной матрицы $C$.
Тогда $\rk B' = \rk B$, так как он не меняется при умножении слева и справа на невырожденную матрицу (утверждение~\ref{claim::rkInvariance}).

\paragraph{След}
Так как $\tr(B') \neq \tr(C^t B C)$, то вообще говоря след не несет никакой содержательной информации.
Действительно, рассмотрите пример $B = \begin{pmatrix}{1}&{0}\\{0}&{-1}\end{pmatrix}$ и $C = \begin{pmatrix}{a}&{b}\\{c}&{d}\end{pmatrix}$ -- невырожденная матрица, то есть $ad - bc\neq 0$.
Тогда $\tr(B) = 0$ и $\tr(B') = (a^2 + b^2) - (c^2 + d^2)$.
В случае поля $\mathbb R$ или $\mathbb C$ это означает, что след $B'$ может быть каким угодно числом.

\paragraph{Определитель}

В этом случае $\det(B') = \det(C^t B C) = \det(B) \det (C)^2$.
То есть определитель в разных базисах может отличаться на квадрат числа из $F^*$.
В общем случае это означает, что корректно определено понятие $\det(B) = 0$ или $\det(B)\neq 0$.
Например:
\begin{itemize}
\item Если поле $F = \mathbb C$, то условие $\det(B) = 0$ или $\det(B)\neq 0$ -- лучшее, что можно получить.
Действительно, если $\det(B)\neq 0$, то в другом базисе $\det(B') = \det(B)c^2$.
Так как в поле $\mathbb C$ из любого числа можно извлечь корень, то $\det(B')$ можно сделать произвольным.

\item Пусть $F = \mathbb R$, тогда корректно определен знак определителя, то есть $\sgn \det B$ корректно определен.
Действительно, $\det(B') = \det(B) c^2$.
В поле $\mathbb R$ число равно квадрату тогда и только тогда, когда оно положительно.

\item Пусть $F = \mathbb Q$.
Тут ничего особенно сказать нельзя, кроме явной формулировки.
Пусть $\det(B) = p_1^{k_1}\ldots p_r^{k_r}$, где $p_i\in \mathbb N$ -- простые числа, а $k_i\in \mathbb Z$ -- степени (вообще говоря положительные или отрицательные).
Тогда четности чисел $k_i$ определены однозначно независимо от базиса.
\end{itemize}

Как мы видим определитель вообще говоря не определен для билинейной формы, но какие-то характеристики из него вытащить можно и эти характеристики сильно зависят от поля, над которым мы работаем.

\paragraph{Обратимость матрицы}
Так как корректно определен ранг или понятие $\det(B) = 0$, то обратимость матрицы $B$ тоже не зависит от базиса.



\paragraph{Матрицы симметричных и кососимметричных форм}

\begin{claim}
\label{claim::BilSymAntiSym}
Пусть $\beta\colon V\times V\to F$ -- билинейная форма.
Тогда
\begin{enumerate}
\item Билинейная форма $\beta$ симметрична тогда и только тогда, когда матрица $B_\beta$ симметрична $B_\beta^t = B_\beta$.

\item Если $2\neq 0$ в поле $F$, то билинейная форма кососимметрична тогда и только тогда, когда матрица $B_\beta$ кососимметрична $B_\beta^t = - B_\beta$.

\item Если $2 = 0$ в поле $F$, то билинейная форма кососимметрична тогда и только тогда, когда матрица $B_\beta$ симметрична $B_\beta^t = B_\beta$ и имеет нулевую диагональ.%
\footnote{
\label{foot::AntiSymMatrix}
Над полем $F$, где $2 = 0$, эти два условия по определению полагаются условием кососимметричности матрицы.
Тогда получается, что билинейная форма кососимметрична тогда и только тогда, когда матрица кососимметрична.}
\end{enumerate}
\end{claim}
\begin{proof}
(1) Фиксируем базис $e_1,\ldots,e_n$, тогда билинейная форма превращается в $\beta(x,y) = x^t B y$.
Условие $\beta(x,y) = \beta(y,x)$ для любых $x,y\in F^n$ равносильно условию $x^t B y = y^t B x = (y^t B x)^t = x^t B^t y$ для любых $x,y\in F^n$.
Заметим, что это условие равносильно симметричности матрицы $B$, которая в свою очередь является матрицей $B_\beta$ в базисе $e_1,\ldots,e_n$.

(2) Если $2\neq 0$ в поле $F$, то кососимметричность равносильна условию $\beta(x,y) = -\beta(y,x)$.
Выберем произвольный базис $e_1,\ldots,e_n$, тогда $\beta(x,y) = x^t By$.
Тогда условие $\beta(x,y) = -\beta(y,x)$ для всех $x,y\in F^n$ равносильно условию $x^t By = -y^t Bx = -(y^t B x)^t =- x^t B^t y$ для любых $x,y\in F^n$.
Последнее равносильно условию $B^t = - B$.

(3) Пусть теперь $2 = 0$ в $F$.
Тогда надо заметить, что $1 + 1 = 0$, то есть $1 = - 1$.
То есть условие $B^t = B$ и $B^t = - B$ равносильны!
Как и выше выберем некоторый базис $e_1,\ldots,e_n$ и в нем запишем нашу форму в виде $\beta(x,y) = x^t B y$.
Если $\beta$ кососимметрична, то $\beta(x,y) = -\beta(y,x) = \beta(y,x)$, то есть форма симметрична.
А значит $B^t = B$ по первому пункту.
С другой стороны, так как $\beta(e_i,e_i) = 0$, то диагональ матрицы $B$ должна быть нулевой.

Обратно.
Этот случай противной математики, когда без счета в лоб не обойтись (сочувствую нам всем).
Пусть $v = x_1e_1+\ldots+x_ne_n$ -- некоторый вектор, посчитаем $\beta(v,v)$ и покажем, что мы получим ноль.
\begin{gather*}
\beta(v,v) = \beta(\sum_i x_i e_i, \sum_j x_j e_j) = \sum_{ij}x_ix_j \beta(e_i,e_j) =\\ \sum_{i=1}^nx_i^2 \beta(e_i, e_i) + \sum_{i<j}(x_i x_j \beta(e_i, e_j) + x_j x_i \beta(e_j, e_i)) = \sum_{i=1}^nx_i^2 \beta(e_i, e_i) + 2\sum_{i<j}x_i x_j \beta(e_i, e_j)
\end{gather*}
Тогда в последней сумме первое слагаемое равно нулю так как диагональ $B_\beta$ состоит из нулей, а второе равно нулю, так как $2 = 0$.
\end{proof}

Как мы видим в случае $2 = 0$ в поле $F$ нас ждал сюрприз.
Самым большим откровением обычно становится тот факт, что кососимметрические билинейные формы в данном случае становятся специального вида симметрическими формами.
Вот такие чудесатые чудеса.
