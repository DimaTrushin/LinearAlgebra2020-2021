\ProvidesFile{lecture09.tex}[Лекция 9]


\subsection{Теорема Гамильтона-Кэли}

\paragraph{Многочлены с матричными коэффициентами}

Обозначим через $\Matrix{n}[t]$ множество многочленов от переменной $t$ имеющих матричные коэффициенты из $\Matrix{n}$, т.е.
\[
\Matrix{n}[t] = \{A_0+A_1t+\ldots + A_k t^k\mid A_i\in \Matrix{n}\}
\]
здесь $t$ -- формальная переменная, которая представляет собой неизвестное число.
Про эти многочлены надо думать как про картинки.
Такие картинки можно складывать и умножать по формулам известным для многочленов с обычными числовыми коэффициентами:
\begin{itemize}
\item Сумма.
\[
\left(\sum_{i}A_i t^i\right) +\left (\sum_{j}B_j t^j\right) = \sum_{i}(A_i+ B_i) t^i
\]

\item Произведение.
\[
\left(\sum_i A_i t^i\right)\left( \sum_j B_j t^j\right) = \sum_k \left(\sum_{s+t = k}A_s B_t\right)t^k
\]
\end{itemize}
Надо лишь отметить, что в произведении нельзя переставлять местами $A_s$ и $B_t$, так как матрицы вообще говоря не перестановочны.

\paragraph{Подстановка матрицы в многочлен} 

Теперь для произвольного многочлена $f\in\Matrix{n}[t]$ и матрицы $D\in\Matrix{n}$ определим подстановку матрицы $D$ в многочлен $f$ справа:
\[
f(D) = A_0 + A_1 D + \ldots + A_k D^k
\]
т.е. мы вместо $t$ подставляем всюду матрицу $D$.
Аналогично, можно определить левую подстановку:
\[
(D)f = A_0 + D A_1 + \ldots + D^k A_k
\]
Надо отметить, что вообще говоря $f(D)\neq (D)f$.
Мы всегда будем пользоваться только правой подстановкой.

\paragraph{Свойства подстановки}

Пусть $f,g\in\Matrix{n}[t]$ -- два многочлена и $D\in\Matrix{n}$ -- некоторая матрица.
Сделаем следующие замечания:
\begin{enumerate}
\item Всегда верно равенство
\[
f(D) + g(D) = (f + g)(D)
\]
\item Для произведения вообще говоря выполнено
\[
f(D)g(D)\neq (fg)(D)
\]
Действительно, возьмем $f(t) = t$, $g(t) = Bt$, тогда $(fg)(t) = Bt^2$.
В этом случае $f(D)g(D) = DBD$, а $(fg)(D) = BDD$.
Вообще говоря, имеем $DBD \neq BDD$ если матрицы $B$ и $D$ не коммутируют.

\item Если $D$ коммутирует со всеми коэффициентами матрицы $g$, то верно равенство
\[
f(D)g(D) = (fg)(D)
\]
Это видно непосредственно из определения умножения и подстановки.
\end{enumerate}

\paragraph{Теорема}

Теперь мы готовы к формулировке и доказательству полезного результата.

\begin{claim}
[Теорема Гамильтона-Кэли]
Пусть $A\in\Matrix{n}$.
Тогда $\chi_A(A) = 0$.
\end{claim}

Прежде чем доказывать теорему, давайте объясним в чем сложность и почему дурацкие доказательства не работают.
Смотрите, у нас $\chi_A(\lambda) = \det(\lambda E - A)$.
Хочется подставить вместо $\lambda$ матрицу $A$ и сказать, что определитель нулевой матрицы равен $0$.
В этом рассуждении есть лажа.
Давайте продемонстрируем ее на матрице $2$ на $2$.
Пусть
\[
A = 
\begin{pmatrix}
{a_{11}}&{a_{12}}\\
{a_{21}}&{a_{22}}\\
\end{pmatrix}
\]
Тогда
\[
\det(A-\lambda E) =
\det
\left(
\begin{pmatrix}
{a_{11}}&{a_{12}}\\
{a_{21}}&{a_{22}}\\
\end{pmatrix}
-\lambda E
\right)=
\det 
\begin{pmatrix}
{a_{11}-\lambda}&{a_{12}}\\
{a_{21}}&{a_{22}-\lambda}\\
\end{pmatrix}
\]
Так вот, последнее равенство верно если $\lambda$ является числом.
Если же $\lambda$ является матрицей, то оно непонятно, что значит.
Можно понимать правую часть как блочную матрицу $2$ на $2$ из блоков $2$ на $2$ (т.е. всего $4$ на $4$), но тогда это просто не верное равенство.
Это рассуждение можно докрутить до верного, но тогда в правой части надо использовать вместо определителя более хитрую его версию.
Подобное рассуждение растет из коммутативной алгебры, где доказательство естественным образом сводится к формулам Крамера, но для его освоения надо знать, что такое кольца и модули.
Мы же пойдем чуть более простым путем.

\begin{proof}
Рассмотрим матрицу $\lambda E - A$, где $\lambda$ -- неизвестное число.
Введем следующее обозначение $R(\lambda) = \widehat{\lambda E - A}$.

Заметим, что каждый коэффициент $R(\lambda)$ является многочленом от $\lambda$, т.е. $R(\lambda) = (r_{ij}(\lambda))$ и $r_{ij}(\lambda)$ -- многочлен.
То есть $r_{ij}(\lambda) = \sum_k r_{ijk}\lambda^k$.
Тогда $R(\lambda) = \sum_k R_k \lambda^k$, где $R_k=(r_{ijk})$.
То есть $R(\lambda)\in\Matrix{n}[\lambda]$.

Теперь применим формулы для перемножения матрицы с ее присоединенной из утверждения~\ref{claim::InvMatExplicite} для матрицы $\lambda E - A$, получим 
\[
(\lambda E - A)R(\lambda) = R(\lambda)(\lambda E - A) = \det(\lambda E - A)E=\chi_A(\lambda)E
\]
Нас интересует только равенство
\[
R(\lambda)(\lambda E - A) = \chi_A(\lambda)E
\]
Тогда рассмотрим многочлены $f(\lambda) = R(\lambda)$, $g(\lambda) = \lambda E  - A$.
В этом случае $(fg)(\lambda) = \chi_A(\lambda)E$.
Возьмем в качестве матрицы $D$ матрицу $A$.
Заметим, что она коммутирует с коэффициентами $g$, потому что это $E$ и $-A$.
Значит верно равенство $f(D)g(D) = (fg)(D)$.
Последнее означает
\[
0 = R(A)(A E - A) = \chi_A(A) E = \chi_A(A)
\]
Что и требовалось доказать.
\end{proof}


\newpage
\section{Комплексные числа}

\subsection{Идея}

Почему нам вдруг не хватает вещественных чисел?
Давайте вспомним, а откуда получились вещественные числа?
Для начала у нас есть натуральные числа: $\mathbb N = \{1,2,3,\ldots\}$.
Которые мы беззаботно складывали и умножали.
Но как только нам захотелось посчитать $5 - 8$, как нам понадобились другие числа -- целые $\mathbb Z = \{\ldots, -2, -1,0,1,2,\ldots\}$.
И мы опять жили долго и счастливо, пока на не пришлось делить $2/3$ и тут пришлось построить рациональные числа $\mathbb Q = \{\frac{p}{q}\mid p,q\in \mathbb Z,\, q\neq 0\}$.
Вещественные нам пригодились, когда надо было решить уравнение $x^2 = 2$.
Тогда пришлось добавить $\sqrt{2}$, а заодно и кучу других полезных чисел.
Однако, этого опять оказалось мало и уравнение $x^2 +1 = 0$ не решается в вещественных числах.
При этом давайте заметим, что добавляя новые числа, операции над старыми мы не меняли.
Мы добавили целые, рациональные, вещественные, а натуральные как складывались и умножались по старым правилам, так и продолжают складываться и умножаться.

А какие-же числа мы хотим получить в идеале.
Прежде всего хочется решать уравнения вида $f(x) = 0$, где $f\in \mathbb R[x]$, всегда, когда это возможно.
Например, если $f = 1$, то решить такое уравнение по понятным причинам не возможно, но вот если $\deg f > 0$, то очень хочется иметь решение.
Новые числа должны содержать все вещественные как подмножество.
Но кроме этого, мы хотим уметь делать все арифметические операции с новыми числами, да еще так, чтобы старые операции не изменились.
И еще хочется по возможности быть экономными.
Вдруг, можно построить много разных лишних чисел (например, когда нам понадобились рациональные числа, мы могли по наивности и безрассудству сразу же построить вещественные, но обошлись более экономным вариантом в виде рациональных чисел).

Для того, чтобы формализовать идеи выше, нам надо строго сказать, а какой математической структурой должны являться новые числа.
Такими структурами являются поля.
Потом надо объяснить как правильно обращаться с полями, что с ними можно делать, как их сравнивать между собой.
И как только у нас появился зверинец полей, мы можем найти в нем поле комплексных чисел, как самое лучшее, которое только возможно среди тех, что удовлетворяют нашим запросам.


Сейчас нас ждет очередное абстрактное определение.
Напомню, что оно всегда состоит из двух частей: в первой части сказано какие у нас данные, а во второй -- каким аксиомам эти данные подчиняются.

\begin{definition}
[Поле]
Поле это следующий набор данных: $(F, + , \cdot)$, где
\begin{itemize}
\item $F$ -- некоторое множество.
Элементы этого множества называются числами.

\item $+\colon F\times F\to F$, $(x,y)\mapsto x+y$ -- некоторая операция называемая сложением.

\item $\cdot\colon F\times F\to F$, $(x,y)\mapsto xy$ -- некоторая операция называемая умножением.
\end{itemize}
Эти данные должны подчиняться следующим десяти аксиомам:
\begin{enumerate}
\item {\bf Ассоциативность сложения}
Для любых элементов $x,y,z\in F$ выполнено $x+(y+z) = (x+y)+z$.

\item {\bf Существования нейтрального по сложению}
Существует такой элемент $0\in F$ такой, что для любого $x\in F$ верно $x + 0 = 0 + x = x$.
Такой элемент называется нулем.

\item {\bf Существование обратного по сложению}
Для любого $x\in F$ существует элемент $-x\in F$ такой, что $x + (-x) = (-x) + x = 0$.
Такой элемент называется противоположным.

\item {\bf Коммутативность сложения}
Для любых элементов $x,y\in F$ верно $x+y = y+x$.

\item {\bf Ассоциативность умножения}
Для любых элементов $x,y,z\in F$ верно $x(yz) = (xy)z$.

\item {\bf Существование нейтрального по умножению}
Существует такой элемент $1\in F$, что для любого $x\in F$, верно $x 1 = 1 x = x$.
Такой элемент называется единицей.

\item {\bf Существование обратного по умножению}
Для любого элемента $x\in F\setminus\{0\}$ существует элемент $x^{-1}\in F$ такой, что $x x^{-1} = x^{-1}x = 1$.
Такой элемент называется обратным к $x$.

\item {\bf Коммутативность умножения}
Для любых элементов $x,y\in F$ верно $xy = yx$.

\item {\bf Дистрибутивность}
Для любых элементов $x,y,z\in F$ верно $x(y+z) = xy + xz$ и $(x+y)z = xz + yz$.

\item {\bf Нетривиальность}
$0\neq 1$.
\end{enumerate}
\end{definition}

\paragraph{Замечания}

Давайте сделаем несколько полезных замечаний.
\begin{enumerate}
\item Аксиомы сгруппированы следующим образом: (1--4) аксиомы на сложение, (5--8) аксиомы на умножение, (9) связь между сложением и умножением, (10) нетривиальность.
Причем аксиомы (1--4) и (5--8) идут по одному и тому же шаблону: ассоциативность, нейтральный элемент, обратный, коммутативность.
НО стоит отметить важную разницу между аксиомами (3) и (7).
По сложению обратный должен быть для любого элемента, по умножению только для ненулевого.
В частности, аксиому (7) нельзя сформулировать без аксиомы (2).

\item В аксиомах (2) и (6) не требуется единственность нуля и единицы.
Однако, можно показать, что если ноль существует, то он обязательно единственный, аналогично с единицей.
Действительно, если у нас есть два нуля $0_1$ и $0_2$, то рассмотрим их сумму $0_1 + 0_2$.
Так как $0_1$ является нулем, то $0_1 + 0_2 = 0_2$.
Так как $0_2$ является нулем, то $0_1 + 0_2 = 0_1$.
Значит оба нуля совпадают.
Аналогично проверяется единственность единицы.
Потому в силу однозначности эти элементы обозначаются $0$ и $1$.

\item В аксиомах (3) и (7) не требуется единственность обратного.
Однако, можно показать, что для любого $x$ существует единственный $-x$ и единственный $x^{-1}$.
Действительно, если для элемента $x\in F$ есть два элемента $y,z\in F$ таких, что
\[
x + y = y + x = 0\quad\text{и}\quad x + z = z + x = 0
\]
Тогда рассмотрим выражение
\[
(y + x) + z = y + (x + z)
\]
Его левая часть вычисляется в $0 + z = z$, а правая вычисляется в $y + 0  = y$.
А значит $y$ и $z$ совпадают.
То есть обратный по сложению будет один, аналогично с обратным по умножению.
Именно по причине однозначности им даются такие имена.
В частности однозначно определено число $-1$.

\item Мы привыкли к всяким замечательным свойствам, которым подчиняются числа $0$ и $1$.
Например: $x 0 = 0$ или $(-1)x = -x$ для любого $x$.
Оказывается, что их можно доказать пользуясь аксиомами.
Попробуйте сделать это.

\item Давайте рассмотрим множество $F=\{\cdot\}$ состоящее из одной точки.
Тогда на таком множестве существует единственная операция, положим сложение и умножение равными ей.
Тогда данный набор данных удовлетворяет всем аксиомам поля кроме последней.
Здесь ноль равен единице и вообще все элементы равны друг другу и ничего кроме  нуля (единицы) у нас  нет.
На самом деле, если ноль равен единице, то никаких других структур мы не построим.
Действительно, предположим $(F, +, \cdot)$ удовлетворяет всем аксиомам с первой по девятую, а вместо десятой -- ее отрицание $1 = 0$.
Тогда для любого элемента $a\in F$ имеем $a = a \cdot 1 = a \cdot 0$.
Давайте покажем, что $a \cdot 0 = 0$ для любого $a\in F$.
Рассмотрим равенство $0 + 0 = 0$, которое следует из определения нуля.
Умножим его на $a$, получим $a\cdot (0 + 0) = a \cdot 0$.
Раскроем скобки и получим $a \cdot 0 + a\cdot 0 = a\cdot 0$.
Теперь прибавим к обеим частям равенства элемент $- (a\cdot 0)$.
Получим
\[
a \cdot 0 + a\cdot 0 + - (a\cdot 0) = a\cdot 0 + - (a\cdot 0)
\]
Что равносильно $a\cdot 0 + 0 = 0$, а значит $a\cdot 0 = 0$, что и требовалось.
Потому последняя аксиома нужна для того, чтобы исключить именно этот дурацкий пример.
\end{enumerate}

\paragraph{Примеры}

Как только вам скормили абстрактное определение, первым делом нужны примеры.
Он помогут по-новому взглянуть на старых знакомых и разобраться с тем, а как вообще задавать эти самые новые объекты.

\begin{enumerate}
\item Рациональные и вещественные числа $\mathbb Q$ и $\mathbb R$ с обычными операциями.

\item Множество $\mathbb Q[\sqrt{2}] = \{a + b \sqrt{2}\mid a,b\in\mathbb Q\}\subseteq \mathbb R$ с обычными операциями является примером поля, которое лежит между $\mathbb Q$ и $\mathbb R$.

\item Рациональные функции $\mathbb R(x) = \{\frac{f}{g}\mid f,g\in\mathbb R[x]\}$.
Давайте думать про рациональные функции как про картинки.
Тогда на них определены формальные операции сложения и умножения.
Относительно этих операций они являются полем.

\item Теперь время экзотики $\mathbb F_2 =\{0, 1\}$, а операции берутся по модулю $2$.
Можно проверить, что и этот товарищ является полем.
Это очень важное поле для computer science.
Оно и его аналоги используются в теории кодирования, восстановления сигнала, архивирования и т.д.
\end{enumerate}
