\ProvidesFile{lecture30.tex}[Лекция 30]


\begin{definition}
Пусть $V$ -- евклидово пространство с фиксированной ориентацией.
Тогда определим ориентированный $n$-мерный объем следующим образом.
Пусть $(v_1,\ldots,v_n)\in V$, тогда
\[
\Vol^{or}_n\Pi(v_1,\ldots,v_n) = 
\left\{
\begin{aligned}
0&,& &v_1,\ldots,v_n\text{ линейно зависимы}\\
\Vol_n\Pi(v_1,\ldots,v_n)&,& &(v_1,\ldots,v_n)\text{ положительно ориентирован}\\
-\Vol_n\Pi(v_1,\ldots,v_n)&,& &(v_1,\ldots,v_n)\text{ отрицательно ориентирован}
\end{aligned}
\right.
\]
\end{definition}


\begin{claim}
Пусть $V$ -- евклидово пространство размерности $n$ с фиксированной ориентацией, $(v_1,\ldots,v_n)\in V^n$ и $C\in \Matrix{n}$.
Тогда
\[
\Vol^{or}_n\Pi((v_1,\ldots,v_n)C) = \det C \Vol^{or}_n\Pi(v_1,\ldots,v_n)
\]
\end{claim}
\begin{proof}
Если матрица $C$ вырождена, то набор векторов $(v_1,\ldots,v_n)C$ линейно зависим, а значит левый объем равен  нулю.
С другой стороны $\det C = 0$, а значит и правая часть равна нулю.
Аналогично, если $(v_1,\ldots,v_n)$ линейно зависимый набор, то и набор $(v_1,\ldots,v_n)C$ тоже линейно зависим.
А тогда объемы в левой и правой частях равны нулю.
Осталось разобраться со случаем $(v_1,\ldots, v_n)$ -- базис и $C$ -- невырожденная матрица.
В этом случае совпадение левой и правой части по модулю следует из утверждения~\ref{claim::Volume} об изменении объемов при линейной замене образующих параллелепипеда.
Осталось проверить, что у них совпадают знаки.
Если $\det C > 0$, то наборы $(v_1,\ldots,v_n)$ и $(v_1,\ldots, v_n)C$ одинаково ориентированы.
Значит знаки у $\Vol^{or}_n\Pi((v_1,\ldots, v_n)C) $ и $\Vol^{or}_n \Pi((v_1,\ldots, v_n))$ одинаковые, а к тому же $\det C > 0$.
Значит знаки обеих частей равны.
Пусть теперь $\det C < 0$.
Тогда знаки у $\Vol^{or}_n\Pi((v_1,\ldots, v_n)C) $ и $\Vol^{or}_n \Pi((v_1,\ldots, v_n))$ разные.
Но при этом $\det C < 0$, что делает знаки левой и правой части одинаковыми.
Победа!
\end{proof}

\paragraph{Пример}

Пусть $V = \mathbb R^n$ со стандартным скалярным произведением $(x, y) = x^t y$.
И пусть ориентация зафиксирована так, что стандартный базис является положительным.
Возьмем $v_1,\ldots,v_n\in \mathbb R^n$ произвольный набор векторов.
Образуем матрицу $A = (v_1|\ldots|v_n)\in \Matrix{n}$.
Тогда 
\begin{itemize}
\item $G(v_1,\ldots,v_n) = A^t A$.

\item $\det G(v_1,\ldots,v_n) = \det A^2$.

\item $\Vol_n\Pi(v_1,\ldots,v_n) = |\det A|$.

\item $\Vol^{or}_n\Pi(v_1,\ldots,v_n) = \det A$.
\end{itemize}
Заметьте, что ориентация набора (как и знак соответствующего объема) меняется при перестановке векторов в наборе на знак совершенной перестановки.
Другая причина знака объема -- смена направления вектора, то есть когда вектор $v_i$ в наборе меняется на вектор $-v_i$.

\begin{claim}
Пусть $V$ -- ориентированное евклидово пространство размерности $n$, $v_1,\ldots,v_n\in V$ -- некоторый набор векторов и $\varphi\colon V\to V$ -- линейный оператор.
Тогда 
\[
\Vol^{or}_n(\varphi(\Pi(v_1,\ldots,v_n))) = \det \varphi \Vol^{or}_n(\Pi(v_1,\ldots,v_n))
\]
\end{claim}
\begin{proof}
Выберем положительно ориентированный ортонормированный базис $e_1,\ldots,e_n$.
Тогда $V$ превращается в $\mathbb R^n$, скалярное произведение становится стандартным $(x, y) = x^t y$, линейное отображение превращается в умножение на матрицу $\varphi(x) = Cx$, а векторы $v_1,\ldots,v_n$ расставим по столбцам матрицы $A = (v_1|\ldots|v_n)$.

Как мы видели в предыдущем примере
\[
\Vol^{or}_n(\Pi(v_1,\ldots,v_n)) = \det A,\quad
\Vol^{or}_n(\varphi(\Pi(v_1,\ldots,v_n))) = \det CA
\]
А так как $\det C = \det \varphi$ по определению, утверждение вытекает из мультипликативности определителя.
\end{proof}


\newpage
\section{Комплексные векторные пространства}

В начале несколько общих слов о том, зачем все это надо и куда оно нас заведет.
Этот раздел будет полностью посвящен векторным пространствам над полем $\mathbb C$.
Основная задача этого поля -- помогать решать трудности поля $\mathbb R$.
Но для этого нам надо уметь заменять вещественные объекты комплексными.
Например, вещественное векторное пространство превращать в комплексное векторное пространство.
Беда с евклидовыми пространствами в том, что среди билинейных форм над $\mathbb C$ нет аналога скалярного произведения.
Потому приходится от билинейных форм переходить к так называемым полуторалинейным.
Оказывается, что в этом случае можно построить полноценный аналог евклидовых пространств в комплексном мире, такие пространства называются эрмитовыми (но я иногда буду их называть евклидовыми над $\mathbb C$, чтобы подчеркнуть аналогию).
Кроме этого я покажу как переходить от вещественных объектов к комплексным и наоборот.
Мы будем менять пространства, операторы, билинейные формы и прочее.
Основная идея будет уследить за тем, как при подобных заменах меняются характеристики этих объектов.
Этому будет посвящен раздел комплексификации и овеществления.

\subsection{Полуторалинейные формы}

\begin{definition}
Пусть $V$ -- векторное пространство над полем $\mathbb C$, отображение $\beta \colon V\times V\to \mathbb C$ называется полуторалинейным, если выполнены следующие свойства:
\begin{enumerate}
\item $\beta(v_1+v_2, u) = \beta(v_1,u) + \beta(v_2, u)$, для любых $v_1, v_2, u\in V$

\item $\beta(\lambda v, u) = \bar \lambda \beta(v, u)$, для любых $v, u \in V$ и $\lambda\in \mathbb C$.

\item $\beta(v, u_1 + u_2) = \beta(v, u_1) + \beta(v, u_2)$, для любых $v, u_1, u_2\in V$.

\item $\beta(v, \lambda u) = \lambda\beta(v, u)$, для любых $v,u \in V$ и $\lambda \in \mathbb C$.
\end{enumerate}
\end{definition}

В этом случае говорят, что $\beta$ полулинейна по первому аргументу и линейна по второму.
Обратите внимание, что выбор первого аргумента для полулинейности является случайным и в разной литературе принято по-разному определять полуторалинейность.
В одних источниках полулинейность в первом аргументе, в других -- во втором.

\begin{definition}
Пусть $V$ -- векторное пространство над $\mathbb C$.
Множество всех полуторалинейных форм на пространстве $V$ я буду обозначать через $\Bil_{1\frac{1}{2}}(V)$.
Это множество является векторным пространством над $\mathbb C$ относительно операций:
\[
(\beta_1+\beta_2)(v, u) := \beta_1(v,u) + \beta_2(v,u)\quad\text{и}\quad (\lambda\beta)(v,u) = \lambda\beta(v,u),\quad\text{где}\quad v,u\in V,\;\lambda\in \mathbb C
\]
\end{definition}

Для удобства введем следующее техническое определение.
\begin{definition}
Пусть $A\in \operatorname{M}_{m\,n}(\mathbb C)$, тогда определим матрицу $\bar A$ как матрицу, в которой мы применили сопряжение ко всем элементам матрицы $A$, то есть $(\bar A)_{ij} = \overline{A_{ij}}$.
Кроме того, определим $A^* = \bar A^t$ и будем называть матрицу $A^*$ эрмитово сопряженной к матрице $A$.%
\footnote{Вообще говоря, правильный подход к определению матрицы $A^*$ следующий.
В случае вещественного поля надо положить $A^* = A^t$, а в случае комплексного $A^* = \bar A^t$.
Тогда у нас одно обозначение и надо лишь упоминать в каком мире мы живем -- комплексном или вещественном.
Я предпочитаю единую технику, которая работает в разных ситуациях, чем иметь миллион разных способов под каждую конкретную задачу.}
\end{definition}

Как обычно начинаем с конструкторов и правил, как работать с новыми объектами.

\begin{claim}
Пусть $V$ -- векторное пространство над $\mathbb C$ и $\beta \colon V\times V \to \mathbb C$ -- полуторалинейная форма.
Тогда
\begin{enumerate}
\item Для любого базиса $e_1,\ldots,e_n\in V$ определим матрицу $B$ с коэффициентами $\beta(e_i, e_j)$.
Тогда в координатах базиса $e_1,\ldots,e_n$ полуторалинейная форма записывается в виде $\beta(x, y) = \bar x^t B y$.

\item Если $(e'_1,\ldots,e'_n) = (e_1,\ldots,e_n)C$ -- некоторый другой базис, в котором матрица полуторалинейной формы есть $B'$ с коэффициентами $\beta(e'_i,e'_j)$, то $B' = \bar C^t B C = C^* B C$.

\item Для любого фиксированного базиса $e_1,\ldots,e_n\in V$ отображение $\Bil_{1\frac{1}{2}}(V)\to \operatorname{M}_n(\mathbb C)$ сопоставляющее полуторалинейной форме $\beta$ ее матрицу $B = (\beta(e_i,e_j))$ является изоморфизмом векторных пространств.
\end{enumerate}
\end{claim}
\begin{proof}
(1) Если $e = (e_1,\ldots,e_n)$ -- базис в $V$ и $v,u\in V$, тогда $v = ex$ и $u = ey$, где $x,y\in \mathbb C^n$.
В этом случае
\[
\beta(v,u) = \beta(\sum_i x_i e_i, \sum_j y_j e_j) = \sum_{ij}\bar x_i y_j \beta(e_i, e_j) = \bar x^t B y
\]

(2) Пусть в этом случае $v,u\in V$ и при этом $v = ex = e' x'$ и $u = ey = e'y'$, где $x,y,x',y'\in \mathbb C^n$.
Тогда по формулам замены координат вектора (формулы в конце раздела~\ref{subsection::FnSpace}) $x = Cx'$ и $y = Cy'$.
Тогда
\[
\beta(v,u) =\beta(x,y) = \bar x^t B y = \overline{Cx'}^t B C y' = (\bar x')^t \bar C^t B C y' = \beta(x',y') = (\bar x')^t B' y'
\]
Значит $B' = \bar C^t B C = C^* B C$.

(3) Линейность правила $\beta\mapsto B = (\beta(e_i, e_j))$ проверяется непосредственно.
Также непосредственно проверяется, что правило $B\mapsto \beta(x,y) = \bar x^t B y$ линейно и задает обратное отображение.
Я оставлю эту рутину на радость читателю.
\end{proof}


\subsection{Сведение к билинейным формам}
Теперь нам хотелось бы обобщить на случай полуторалинейных форм все те факты, что мы доказали для билинейных форм.
Есть два пути: мучительный и долгий, когда мы по аналогии с билинейными формами все доказываем заново, или мучительный и быстрый, когда мы вводим некую неприятную абстракцию, которая объяснит, как свести полуторалинейные формы к билинейным.
Я считаю, что если уж и мучиться, то лучше по-быстрому, и пойду вторым путем.

\begin{definition}
Пусть $V$ -- векторное пространство над $\mathbb C$.
Определим новое векторное пространство $\bar V$ следующим образом.
Как множество $\bar V = V$.
Теперь на нем надо задать операции сложения и умножения на скаляр.
Сложение зададим так же, как оно было задано на $V$.
А умножение определим по формуле $\lambda *v = \bar \lambda v$, где справа стоит исходная операция умножения на скаляр в $V$.
\end{definition}

Обратите внимание, что пространства $V$ и $\bar V$ имеют одинаковую размерность.
Более того, набор векторов $e_1,\ldots,e_n$ является базисом в $V$ тогда и только тогда, когда он является базисом в $\bar V$.

\paragraph{Пример}

Пусть $V = \mathbb C^n$, тогда в пространстве $\bar V$ операции заданы следующим образом
\[
\begin{pmatrix}
{x_1}\\{\vdots}\\{x_n}
\end{pmatrix}
+
\begin{pmatrix}
{y_1}\\{\vdots}\\{y_n}
\end{pmatrix}
=
\begin{pmatrix}
{x_1 + y_1}\\{\vdots}\\{x_n + y_n}
\end{pmatrix}
\quad\text{и}\quad
\lambda
\begin{pmatrix}
{x_1}\\{\vdots}\\{x_n}
\end{pmatrix}
=
\begin{pmatrix}
{\bar \lambda x_1}\\{\vdots}\\{\bar \lambda x_n}
\end{pmatrix}
\]

\paragraph{Замечания}

\begin{itemize}
\item Пусть $\beta\colon V\times V\to \mathbb C$ -- полуторалинейная форма, тогда на нее можно посмотреть как на отображение $\beta\colon \bar V\times V\to \mathbb C$ в этом случае $\beta$ становится билинейной формой.
И наоборот если $\beta\colon \bar V\times V\to \mathbb C$ билинейная форма, то $\beta\colon V\times V\to \mathbb C$ -- полуторалинейная форма.
Таким образом можно пользоваться всеми фактами про билинейные формы, смотря на полуторалинейную форму, как на отображение $\beta\colon \bar V\times V\to \mathbb C$.

\item Заметим, что подмножество $U\subseteq V$ является подпространством в $V$ тогда и только тогда, когда оно является подпространством в $\bar V$.
То есть запас подпространств в $V$ и $\bar V$ одинаковый.
\end{itemize}

\begin{definition}
Пусть $\beta\colon V\times V\to \mathbb C$ -- полуторалинейная форма.
Тогда
\begin{itemize}
\item Для любого подпространства $W\subseteq V$ определим его левое и правое ортогональное дополнение как левое и правое ортогональное дополнение для соответствующей билинейной формы, то есть:
\[
W^\bot = \{v\in V\mid \beta(W,v) = 0\}\quad \text{и} \quad {}^\bot W = \{v\in V\mid \beta(v, W) = 0\}
\]

\item  определим ее левое и правое ядра, как левые и правые ядра соответствующей билинейной формы $\beta\colon \bar V\times V\to \mathbb C$, то есть $\ker^L \beta= {}^\bot V$ и $\ker^R \beta = V^\bot$.

\item Полуторалинейная форма $\beta$ будет называться невырожденной, если соответствующая билинейная форма невырождена.
То есть когда $\ker^L\beta = \ker ^R\beta = 0$.%
\footnote{Так как $\dim \bar V = \dim V$, то это условие равносильно тому, что одно из ядер равно нулю.}
Это равносильно тому, что матрица $B = (\beta(e_i, e_j))$ формы в любом базисе $e_1,\ldots,e_n$ невырождена.
\end{itemize}
 
\end{definition}

\begin{claim}
Пусть $V$ -- векторное пространство над $\mathbb C$ и $\beta \colon V\times V\to \mathbb C$ -- невырожденная полуторалинейная форма.
Тогда
\begin{enumerate}
\item Для любого подпространства $W\subseteq V$ выполнено
\[
\dim W^\bot + \dim W = \dim V
\]

\item Для любого подпространства $W\subseteq V$ выполнено ${}^\bot(W^\bot) = W$.

\item Для любых подпространств $W\subseteq E\subseteq V$ верно, что $W^\bot \supseteq E^\bot$.
Причем $W = E$ тогда и только тогда, когда $W^\bot = E^\bot$.

\item Для любых подпространств $W, E\subseteq V$ выполнено равенство
\[
(W + E)^\bot = W^\bot \cap E^\bot
\]

\item Для любых подпространств $W, E\subseteq V$ выполнено равенство
\[
(W\cap E)^\bot = W^\bot + E^\bot
\]
\end{enumerate}
Аналогичные свойства выполняются для левых ортогональных дополнений ${}^\bot W$.
\end{claim}
\begin{proof}
Это следует из двойственности для подпространств для билинейных форм (утверждение~\ref{claim::DualitySpaces}) и предыдущего замечания.
\end{proof}

\subsection{Квадратичные формы}


\begin{definition}
Пусть $\beta\colon V\times V\to \mathbb C$ -- полутора линейная форма.
Тогда $Q_\beta\colon V\to \mathbb C$ по правилу $v\mapsto \beta(v,v)$ называется квадратичной формой.%
\footnote{Мы можем рассматривать полуторалинейную форму $\beta$, как билинейную форму $\beta\colon \bar V\times V\to \mathbb C$.
Но билинейная форма получается на разных пространствах, потому у нас нет понятия квадратичной формы для $\beta$, как для билинейной формы.
Потому введенное определение квадратичной формы не должно вызвать путаницы.}
\end{definition}

\paragraph{Замечания}

\begin{itemize}
\item Если $e_1,\ldots,e_n\in V$ -- некоторый базис, в котором полуторалинейная форма задается в виде $\beta(x, y) = \bar x^t B y$, то соответствующая ей квадратичная форма имеет вид $Q_\beta(x) = \bar x^t B x$.
Множество квадратичных форм для полуторалинейных форм на пространстве $V$ будем обозначать через $\Quad_{1\frac{1}{2}}(V)$.%
\footnote{Индекс в виде $1\frac{1}{2}$ нужен, чтобы отличить, мы строим квадратичные формы по билинейным или полуторалинейным.}

\item Ключевое свойство квадратичных форм $Q\colon V\to \mathbb C$ заключается в следующем: для любого $v\in V$ и любого $\lambda\in \mathbb C$ выполнено $Q(\lambda v) = |\lambda|^2Q(v)$.
\end{itemize}

Главное отличие от билинейных форм -- любая полуторалинейная форма (без каких-либо дополнительных требований) восстанавливается по своей квадратичной форме.
Давайте разберемся почему это так.

\begin{claim}
[Поляризационная формула]
\label{claim::CPolarization}
Пусть $\beta\colon V\times V\to \mathbb C$ -- произвольная полуторалинейная форма и $Q_\beta\colon V\to \mathbb C$ -- соответствующая ей квадратичная форма.
Тогда 
\[
\beta(v, u) = \sum_{k=0}^3 \frac{i^k}{4}Q_\beta\left(i^k v + u\right) = \frac{Q_\beta(v+u) + iQ_\beta(iv+u) - Q_\beta(-v + u) -i Q_\beta(-iv+u)}{4}
\]
\end{claim}
\begin{proof}
Рассмотрим следующее выражение
\begin{gather*}
\beta(v + u, v+ u) + i\beta(iv + u, iv+ u) - \beta(-v + u, -v+ u) -i \beta(-iv + u, -iv+ u)=\\
\beta(v,v)+\beta(v,u)+\beta(u,v)+\beta(u,u)+\\
+i\beta(v,v)+\beta(v,u)-\beta(u,v)+i\beta(u,u)-\\
-\beta(v,v)+\beta(v,u)+\beta(u,v)-\beta(u,u)-\\
-i\beta(v,v)+\beta(v,u)-\beta(u,v)-i\beta(u,u)=\\
4 \beta(v, u)
\end{gather*}
Разделив обе части на $4$, получаем требуемое.
\end{proof}

\paragraph{Замечание}

В частности, если $\beta(v, v) = 0$ для любого вектора $v\in V$, то полуторалинейная форма $\beta$ тождественно равна нулю.

\begin{claim}
\label{claim::CBilQuad}
Пусть $V$ векторное пространство над $\mathbb C$, тогда отображения
\begin{itemize}
\item $\phi\colon\Bil_{1\frac{1}{2}}(V)\to \Quad_{1\frac{1}{2}}(V)$ по правилу $\beta\mapsto Q_\beta(v) = \beta(v,v)$ (в координатах $\bar x^t B y$ переходит в $\bar x^t B x$)

\item $\psi\colon\Quad_{1\frac{1}{2}}(V)\to \Bil_{1\frac{1}{2}}(V)$ по поляризационной формуле $Q\mapsto \beta_Q(v,u) = \sum_{k=0}^3 \frac{i^k}{4}Q(i^k v + u)$ (в координатах $\bar x^t A x$ переходит в $\bar x^t A y$)
\end{itemize}
являются взаимно обратными изоморфизмами.
\end{claim}
\begin{proof}
По определению, отображение $\phi$ сюръективно.
Из поляризационной формулы (утверждение~\ref{claim::CPolarization}) следует, что композиция $\psi\phi$ равна тождественному отображению.
А значит $\phi$ инъективно, то есть биекция.
При этом $\psi$ является левым обратным к обратимому отображению, а значит просто обратным.
Получили, что $\phi$ и $\psi$ взаимнообратные биекции.
Тот факт, что они изоморфизмы, то есть линейны, непосредственно следует из определения.
\end{proof}

\paragraph{Замечания}

\begin{itemize}
\item Последнее утверждение означает, что полуторалинейные формы и квадратичные формы -- это одно и то же.
В частности, матрица $A$ в определении квадратичной формы однозначно определена.
Действительно, если $Q(x) = \bar x^t A x$ в координатах, то ей соответствует билинейная форма $\beta_Q(x, y) = \bar x^t A y$.
Матрица билинейной формы $A$ определена правилом $a_{ij} = \beta_Q(e_i, e_j)$.

\item Для любопытных, давайте в лоб проверим, что $\phi\psi = \Identity$.
Пусть $Q\colon V\to \mathbb C$ -- произвольная квадратичная форма, тогда она переходит в
\begin{gather*}
\frac{Q(v+v) + iQ(iv+v) - Q(-v+v) -i Q(-iv + v)}{4} = \frac{Q(2v) + iQ((1+i)v) - i Q((1-i)v) - Q(0)}{4} =\\ =\frac{|2|^2 Q(v)+i|1+i|^2Q(v) - i|1-i|^2Q(v)}{4} = Q(v)
\end{gather*}
\end{itemize}

\subsection{Эрмитовы и косоэрмитовы формы}

Так исторически сложилось, что аналоги симметричных и кососимметричных полуторалинейных форм не принято называть симметричными и кососимметричными (как бы мне этого ни хотелось), у них есть другие более пристойные (с точки зрения математического сообщества) названия.
Однако, очень важно понимать, что мы сейчас будем заниматься не чем иным, как изучением аналогов симметричных  и кососимметричных форм, потому и все результаты будут очень ожидаемыми.

\begin{definition}
Пусть $\beta\colon V\times V\to \mathbb C$ -- полуторалинейная форма.
Тогда 
\begin{itemize}
\item $\beta$ называется эрмитовой если $\beta(v, u) = \overline{\beta(u,v)}$ для любых $v,u\in V$.
Это аналог симметричных форм.

\item $\beta$ называется косоэрмитовой, если $\beta(v, u) = -\overline{\beta(u,v)}$ для любых $v,u\in V$.
Это аналог кососимметричных форм.
\end{itemize}
\end{definition}

\paragraph{Замечания}

\begin{itemize}
\item Обратите внимание, что множество эрмитовых (или косоэрмитовых) форм НЕ является векторным пространством над $\mathbb C$, но является векторным пространством над $\mathbb R$.
Действительно, если $\beta(v, u) = \overline{\beta(u,v)}$ для любых $v, u \in V$ и $\lambda\in \mathbb C$, то форма $\lambda \beta$ уже не обязательно эрмитова.
Например, если $\lambda = i$, то $\overline{i\beta(v,u)} = -i \beta(u,v)$.
Как видно из этого вычисления, эрмитова форма после умножения на $i$ превращается в косоэрмитову и наоборот.
В этом смысле изучение эрмитовых или косоэрмитовых форм -- это одно и то же.

\item Правильно думать про эрмитовы и косоэрмитовы формы, как про аналоги вещественной и мнимой части (только мнимая часть рассматривается вместе с мнимой единицей).
В данном случае аналогом комплексного сопряжения является операция $\beta(v,u) \mapsto \overline{\beta(u, v)}$.
Тогда неподвижная часть будет аналогом вещественной части, а меняющая знак -- аналогом мнимой домноженной на $i$.
При этом любая полуторалинейная форма представляется единственным способом в виде суммы эрмитовой и косоэрмитовой:
\[
\beta(v,u) = \frac{\beta(v, u) + \overline{\beta(u, v)}}{2} + \frac{\beta(v, u) - \overline{\beta(u, v)}}{2} 
\]
Или в матричной форме
\[
B = \frac{B + B^*}{2} + \frac{B - B^*}{2}
\]
\end{itemize}

\begin{claim}
\label{claim::CSymBil}
Пусть $\beta\colon V\times V\to \mathbb C$ -- полуторалинейная форма.
Тогда
\begin{enumerate}
\item Пусть $e_1,\ldots,e_n$ -- некоторый базис и $B$ -- матриц полуторалинейной формы в этом базисе.
\begin{itemize}
\item Форма $\beta$ эрмитова тогда и только тогда, когда $B^* = B$.

\item Форма $\beta$ косоэрмитова тогда и только тогда, когда $B^* = - B$.
\end{itemize}

\item Форма $\beta$ эрмитова тогда и только тогда, когда $Q_\beta(v) = \beta(v,v)\in\mathbb R$ для любого $v\in V$.

\item Если форма $\beta$ эрмитова или косоэрмитова, то для любого подпространства $W\subseteq V$ верно $W^\bot = {}^\bot W$.
\end{enumerate}
\end{claim}
\begin{proof}
(1) Пусть после выбора базиса в координатах форма имеет вид $\beta(x, y) = \bar x^t B y$.
Тогда эрмитовость равносильна свойству
\[
\bar x^t B y = \overline{\bar y^t B x} = y^t \bar B \bar x = (y^t \bar B \bar x)^t = \bar x^t \bar B^t y = \bar x^t B^* y
\]
Так как это свойство выполнено для любых $x,y\in \mathbb C^n$, то $B = B^*$.
Аналогично показывается для косоэрмитовых форм.

(2) $\Rightarrow$ Пусть $\beta$ эрмитова.
Тогда $\beta(v,v) = \overline{\beta(v,v)}$.
То есть $Q_\beta(v) = \beta(v,v)\in \mathbb R$.

$\Leftarrow$ Наоборот, применим поляризационную формулу, тогда
\begin{gather*}
\beta(v,u) = \frac{Q(v+u) + iQ(iv+u) - Q(-v+u) - iQ(-iv+u)}{4}=\\
\frac{Q(v+u) + iQ(i(v-iu)) - Q(-(v-u)) - iQ(-i(v+iu))}{4}=\\
\frac{Q(v+u) + i|i|^2Q(v-iu) -|-1|^2 Q(v-u) - i|-i|^2Q(v+iu)}{4}=\\
\frac{Q(v+u) + iQ(v-iu) - Q(v-u) - iQ(v+iu)}{4}
\end{gather*}
Если $Q(v)\in \mathbb R$ для любого $v\in V$, то последнее выражение есть
\[
\overline{\frac{Q(v+u) - iQ(v-iu) - Q(v-u) + iQ(v+iu)}{4}} = \overline{\beta(u,v)}
\]
То есть форма эрмитова.

(3) Для эрмитовых форм по определению получаем
\[
W^\bot = \{v\in V\mid \beta(W, v) = 0\} = \{v\in V\mid \overline{\beta(v, W)} = 0\} = \{v\in V\mid \beta(v, W) = 0\} = {}^\bot W
\]
Аналогично для косоэрмитовых.
\end{proof}

\paragraph{Замечание}

Обратите внимание, что в случае полуторалинейных форм, отличительной особенностью эрмитовых является тот факт, что соответствующая им квадратичная форма принимает только вещественные значения.
Именно это явление и позволяет нам проводить с ними те же самые трюки, которые работали с симметричными билинейными формами над полем $\mathbb R$.


\begin{claim}
Пусть $V$ -- векторное пространство над $\mathbb C$ и $\beta\colon V\times V\to \mathbb C$ -- полуторалинейная форма.
Тогда
\begin{enumerate}
\item Если форма $\beta$ эрмитова, то существует базис $e_1,\ldots,e_n$ такой, что матрица формы $B_\beta$ диагональная с вещественными числами на диагонали.

\item Если форма $\beta$ косоэрмитова, то существует базис $e_1,\ldots,e_n$ такой, что матрица формы $B_\beta$ диагональная с чисто мнимыми числами на диагонали.
\end{enumerate}
\end{claim}
\begin{proof}
(1)$\Rightarrow$(2) Если форма $\beta$ косоэрмитова, то $i\beta$ будет эрмитова.
Тогда в каком-то базисе матрица формы $i\beta$ будет диагональна с вещественными числами на диагонали.
То есть матрица формы $\beta$ диагональная с чисто мнимыми числами на диагонали.

(1) Так как на диагонали у эрмитовой формы всегда находятся вещественные числа (утверждение~\ref{claim::CSymBil} пункт~(2)), то достаточно показать, что матрица формы диагонализуется.
Поступать будем так же, как и в случае билинейных форм.
Рассмотрим квадратичную форму $Q_\beta(v)$.
Если она тождественно равна нулю, то $\beta$ тоже тождественно равна нулю (утверждение~\ref{claim::CBilQuad}).
Тогда матрица нулевая и доказывать нечего.
Пусть теперь $v$ -- такой вектор, что $\beta(v,v) = Q_\beta(v) \neq 0$.
Это значит, что $v\notin \langle v\rangle^\bot$, то есть $\langle v\rangle \cap \langle v\rangle^\bot = 0$.
По определению $\langle v\rangle^\bot = \ker \beta(v, {-})$ является ядром ненулевого линейного функционала, то есть имеет размерность $\dim V - 1$.
Значит $V = \langle v\rangle\oplus \langle v\rangle^\bot$.
Теперь берем вектор $v$ в качестве первого базисного вектора $e_1$.
Индукцией по размерности пространства находим ортогональный базис $e_2,\ldots,e_n$ в $\langle v\rangle^\bot$ для формы $\beta|_{\langle v\rangle^\bot}$.
Победа!
\end{proof}
