\ProvidesFile{lecture06.tex}[Лекция 6]

\newpage

\section{Определитель}

\subsection{Философия}

Сейчас я хочу обсудить <<ориентированный объем>> на прямой, плоскости и в пространстве. 

\paragraph{Прямая}

 На прямой мы можем выбрать <<положительное>> направление.
 Обычно на рисунке выбирают слева направо.
 Тогда длина вектора, который смотрит слева направо, считается положительной, а справа налево -- отрицательной.

\paragraph{Плоскость}

Здесь объем будет задаваться парой векторов, то есть некоторой квадратной матрицей размера $2$, где вектора -- это ее столбцы.
Основная идея такая: пусть мы хотим посчитать площадь между двумя векторами на плоскости, точнее площадь параллелограмма натянутого на вектора $e_1$ и $e_2$ как на первом рисунке ниже.
\[
\xymatrix{
	{}&{}&{}\ar@{--}[r]&{}\\
	{}\ar[urr]^{e_2}\ar[r]_{e_1}&{}\ar@{--}[urr]&{}&{}\\
	{}&{}&{}&{}\\
}\quad
\xymatrix{
	{}&{}&{}\\
	{}\ar[rr]^{e_2}\ar[r]_{e_1}&{}&{}\\
	{}&{}&{}\\
}\quad
\xymatrix{
	{}&{}&{}&{}\\
	{}\ar[drr]_{e_2}\ar[r]^{e_1}&{}\ar@{--}[drr]&{}&{}\\
	{}&{}&{}\ar@{--}[r]&{}\\
}
\]
Давайте двигать вектор $e_2$ к вектору $e_1$.
Тогда площадь будет уменьшаться и когда вектора совпадут, она будет равна нулю.
Однако, если мы продолжим двигать вектор $e_2$, то площадь между векторами опять начнет расти и картинка в конце концов станет симметрична исходной, а полученный параллелограмм равен изначальному.
Однако, эта ситуация отличается от предыдущей и вот как можно понять чем.
Предположим, что между векторами была натянута хорошо сжимаемая ткань, одна сторона которой красная, другая зеленая.
Тогда в самом начале на нас смотрит красная сторона этой ткани, но как только $e_2$ прошел через $e_1$ на нас уже смотрит зеленая сторона.
Мы бы хотели научиться отличать эти две ситуации с помощью знака, если на  нас смотрит красная сторона -- знак положительный, если зеленая -- отрицательный. 

Еще один способ думать про эту ситуацию.
Представим, что плоскость -- это наш стол, а параллелограмм вырезан из бумаги.
Мы можем положить параллелограмм на стол двумя способами: лицевой стороной вверх или же вниз.
В первом случае мы считаем площадь положительной, а во втором -- отрицательной.
Возможность определить лицевую сторону связана с тем, что мы знаем, где у стола верх, а где них.
Это возможно, потому что наша плоскость лежит в трех мерном пространстве и мы можем глядеть на нее извне.
Однако, если бы мы жили на плоскости и у нас не было бы возможности выглянуть за ее пределы, то единственный способ установить <<какой стороной вверх лежит параллелограмм>> был бы с помощью порядка векторов.

Еще одно важное замечание.
Если мы берем два одинаковых параллелограмма на нашем столе, которые лежат лицевой стороной вверх, то мы можем передвинуть один в другой, не отрывая его от стола.
А вот если один из параллелограммов имеет положительный объем, а другой отрицательный, то нельзя перевести один в другой, не отрывая от стола.
То есть, если вы живете на плоскости, то вам не получится переместить положительный параллелограмм в отрицательный, не сломав или не разобрав его.

\paragraph{Пространство}

В пространстве дело с ориентацией обстоит абсолютно аналогично.
Мы хотим уже считать объемы параллелепипедов натянутых на три вектора.
И мы так же хотим, чтобы эти объемы показывали <<с какой стороны>> мы смотрим на параллелепипед.
\[
\xymatrix{
	{}&{}\ar@{--}[rr]\ar@{--}[dl]&{}&{}\ar@{--}[dl]\\
	{}\ar@{--}[rr]&{}&{}&{}\\
	{}&{\cdot}\ar[uu]^(.3){e_3}\ar[rr]^(.3){e_2}\ar[dl]_{e_1}&{}&{}\ar@{--}[dl]\ar@{--}[uu]\\
	{}\ar@{--}[rr]\ar@{--}[uu]&{}&{}\ar@{--}[uu]&{}\\
}
\quad\quad\quad
\xymatrix{
	{}&{}\ar@{--}[rr]\ar@{--}[dl]&{}&{}\ar@{--}[dl]\\
	{}\ar@{--}[rr]&{}&{}&{}\\
	{}&{\cdot}\ar[uu]^(.3){e_2}\ar[rr]^(.3){e_3}\ar[dl]_{e_1}&{}&{}\ar@{--}[dl]\ar@{--}[uu]\\
	{}\ar@{--}[rr]\ar@{--}[uu]&{}&{}\ar@{--}[uu]&{}\\
}
\quad\quad\quad
\xymatrix{
	{}&{}\ar@{--}[rr]\ar@{--}[dl]&{}&{}\ar@{--}[dl]\\
	{}\ar@{--}[rr]&{}&{}&{}\\
	{}&{\cdot}\ar[uu]^(.3){e_1}\ar[rr]^(.3){e_3}\ar[dl]_{e_2}&{}&{}\ar@{--}[dl]\ar@{--}[uu]\\
	{}\ar@{--}[rr]\ar@{--}[uu]&{}&{}\ar@{--}[uu]&{}\\
}
\]
Здесь знак объема определяется по порядку векторов, как знак перестановки.
На рисунке объемы первого и третьего положительные, а у второго отрицательный.
Если вы сделаете модельки этих кубиков из подписанных спичек, то третий кубик -- это первый, но лежащий на другой грани.
А вот второй кубик получить из первого вращениями не получится.
Надо будет его разобрать и присобачить ребра по-другому. 

Как и в случае с плоскостью, если бы мы могли выйти за пределы нашего трехмерного пространства, то у нас появилась бы лицевая и тыльная сторона, как у стола.
И тогда первый и третий кубики лежали бы лицевой стороной вверх, а второй -- вниз.
Мы, конечно же, так сделать не сможем и никогда в жизни не увидим подобное, но думать про такое положение вещей по аналогии с плоскостью можем и эта интуиция бывает полезна.

\paragraph{Пояснение планов}

В текущей лекции я не собираюсь обсуждать объемы, а всего лишь хочу коснуться некоторой техники, которая используется для работы с ориентированными объемами.
Чтобы начать честный рассказ про сами объемы (который обязательно будет, но позже), нам надо поговорить о том, что такое векторное пространство и как в абстрактном векторном пространстве мерить расстояния и углы.
Потому, пока мы не покроем эти темы, всерьез говорить про настоящие объемы мы не сможем.

\subsection{Три разных определения}

\paragraph{Специальные мультипликативные отображения (I)} 

Рассмотрим множество отображений $\psi\colon \Matrix{n}\to \mathbb R$ удовлетворяющие следующими свойствам:
\begin{enumerate}
\item $\psi(AB) = \psi(A)\psi(B)$ для любых $A,B\in\Matrix{n}$.

\item 
$
\psi
\begin{pmatrix}
{1}&{}&{}&{}\\
{}&{\ddots}&{}&{}\\
{}&{}&{1}&{}\\
{}&{}&{}&{d}\\
\end{pmatrix}
= 
d
$ для любого ненулевого $d\in\mathbb R$.
\end{enumerate}
Всюду ниже будем упоминать отображения с такими свойствами, как отображения со свойством (I).%
\footnote{Обратите внимание, что существует много отображений со свойством (1), не удовлетворяющих свойству (2).
Действительно, если $\psi$ -- мультипликативное отображение, то есть удовлетворяет только свойству (1), то $\gamma_n(A) = \psi(A^n)$ -- тоже мультипликативное отображение для любого натурального $n\in \mathbb N$.
Кроме того, $\delta_\alpha(A) = |\psi(A)|^\alpha$ тоже является мультипликативным отображением для любого положительного $\alpha\in \mathbb R$.}

\paragraph{Нормированные полилинейные кососимметрические отображения (II)} 

Пусть $\phi\colon \Matrix{n}\to \mathbb R$ -- некоторое отображение и $A\in\Matrix{n}$.
Тогда про матрицу $A$ можно думать, как про набор из $n$ столбцов: $A = (A_1|\ldots|A_n)$.
Тогда функцию $\phi(A) = \phi(A_1,\ldots,A_n)$ можно рассматривать как функцию от $n$ столбцов. 

В обозначениях выше рассмотрим отображения $\phi\colon \Matrix{n}\to \mathbb R$ удовлетворяющие следующим свойствам:
\begin{enumerate}
\item $\phi(A_1,\ldots, A_i + A_i', \ldots, A_n) = \phi(A_1,\ldots, A_i, \ldots, A_n) + \phi(A_1,\ldots,A_i', \ldots, A_n)$ для любого $i$.

\item $\phi(A_1,\ldots, \lambda A_i, \ldots, A_n) = \lambda \phi(A_1,\ldots, A_i, \ldots, A_n)$ для любого $i$ и любого $\lambda\in\mathbb R$.

\item $\phi(A_1,\ldots, A_i, \ldots, A_j, \ldots, A_n) = -\phi(A_1,\ldots, A_j, \ldots, A_i, \ldots, A_n)$ для любых различных $i$ и $j$.

\item $\phi(E) = 1$.
\end{enumerate}

Первые два свойства вместе называются {\it полилинейностью} $\phi$ по столбцам, т.е. это уважение суммы и умножения на скаляр.
Третье свойство называется {\it кососимметричностью} $\phi$ по столбцам.
Последнее условие -- это условие нормировки.
Данный набор свойств можно заменить эквивалентным с переформулированным третьим свойством:
\begin{enumerate}
\item $\phi(A_1,\ldots, A_i + A_i', \ldots, A_n) = \phi(A_1,\ldots, A_i, \ldots, A_n) + \phi(A_1,\ldots,A_i', \ldots, A_n)$ для любого $i$.

\item $\phi(A_1,\ldots, \lambda A_i, \ldots, A_n) = \lambda \phi(A_1,\ldots, A_i, \ldots, A_n)$ для любого $i$ и любого $\lambda\in\mathbb R$.

\item $\phi(A_1,\ldots, A', \ldots, A', \ldots, A_n) = 0$, т.е. если есть два одинаковых столбца, то значение $\phi$ равно нулю.

\item $\phi(E) = 1$.
\end{enumerate}

Действительно, обозначим $\Phi(a,b) = \phi(A_1,\ldots, a, \ldots, b, \ldots, A_n)$.
Тогда $\Phi$ полилинейная функция двух аргументов.%
\footnote{Такие отображения называются билинейными.}
И нам надо показать, что $\Phi(a,a) = 0$ для любого $a\in \mathbb R^n$ тогда и только тогда, когда $\Phi(a,b) = -\Phi(b,a)$ для любых $a,b\in\mathbb R^n$.
Для $\Rightarrow$ подставим $b = a$, получим $\Phi(a,a) = -\Phi(a,a)$.
Для обратного $\Leftarrow$ подставим $a+b$, получим $\Phi(a+b, a+b) = 0$.
Раскроем скобки: $\Phi(a,a) + \Phi(a,b) + \Phi(b,a) + \Phi(b,b) = 0$.
Откуда следует требуемое.

Везде далее будем упоминать отображения с такими свойствами, как отображения со свойством (II).

\paragraph{Нормированные полилинейные кососимметрические отображения (II')}

Аналогично (II) можно рассмотреть полилинейные кососимметрические отображения по строкам матрицы $A$ вместо столбцов.
Тогда можно рассматривать отображения $\phi'\colon \Matrix{n}\to \mathbb R$ с аналогами четырех свойств выше: полилинейность, кососимметричность, значение $1$ на единичной матрице.
Такие отображения мы будем называть, как отображения со свойствами (II').

\paragraph{Определитель (III)}

Рассмотрим отображение $\det \colon \Matrix{n}\to \mathbb R$ задаваемое следующей формулой: для любой матрицы $A\in\Matrix{n}$ положим
\[
\det A = \sum_{\sigma \in \Sym{n}} \sgn(\sigma) a_{1\sigma(1)}\ldots a_{n\sigma(n)}
\]
Данное отображение называется {\it определителем}, а его значение $\det A$ на матрице $A$ называется определителем матрицы $A$.

Давайте неформально обсудим, как считается выражение для определителя.
Как мы видим определитель состоит из суммы некоторых произведений.
Каждое произведение имеет вид $a_{1\sigma(1)}\ldots a_{n\sigma(n)}$ умноженное на $\sgn(\sigma)$.
Здесь из каждой строки матрицы $A$%
\footnote{Первый индекс -- индекс строки.}
выбирается по одному элементу так, что никакие два элемента не лежат в одном столбце (это гарантированно тем, что $\sigma$ -- перестановка и потому $\sigma(i)$ не повторяются).
Заметим, что слагаемых ровно столько, сколько перестановок -- $n!$ штук.
Из этих слагаемых половина идет со знаком плюс, а другая -- со знаком минус.

\paragraph{План дальнейших действий} 

Наша задача показать, что, во-первых, определитель обладает свойствами (I), (II) и (II'), а, во-вторых, что кроме определителя никакое другое отображение не удовлетворяет этим свойствам.


\subsection{Явные формулы для определителя}

\paragraph{Подсчет в малых размерностях}

\begin{enumerate}
\item Если $A\in \Matrix{1} = \mathbb R$, то $\det A = A$.

\item Если $A \in \Matrix{2}$ имеет вид 
$
A = \left(\begin{smallmatrix}{a}&{b}\\{c}&{d}\end{smallmatrix}\right)
$
, то $\det A = ad - bc$.
Графически: главная диагональ минус побочная.

\item Если $A\in \Matrix{3}$ имеет вид
$
A = \left(\begin{smallmatrix}{a_{11}}&{a_{12}}&{a_{13}}\\{a_{21}}&{a_{22}}&{a_{23}}\\{a_{31}}&{a_{32}}&{a_{33}}\end{smallmatrix}\right)
$
, то определитель получается из $6$ слагаемых три из них с $+$ три с $-$.
Графически слагаемые можно изобразить так:
\[
\det A = 
+
\left(
\parbox{6pt}{
\xymatrix@R=6pt@C=6pt{
	{}\ar@{-}[ddrr]&{}&{}\\
	{}&{}&{}\\
	{}&{}&{}\\
}}
\right) +
\left(
\parbox{6pt}{
\xymatrix@R=6pt@C=6pt{
	{}&{}\ar@{-}[dr]&{}\\
	{}&{}&{}\ar@{-}[dll]\\
	{}\ar@{-}[uur]&{}&{}\\
}}
\right) +
\left(
\parbox{6pt}{
\xymatrix@R=6pt@C=6pt{
	{}&{}&{}\ar@{-}[ddl]\\
	{}\ar@{-}[urr]&{}&{}\\
	{}&{}\ar@{-}[ul]&{}\\
}}
\right) -
\left(
\parbox{6pt}{
\xymatrix@R=6pt@C=6pt{
	{}&{}&{}\ar@{-}[ddll]\\
	{}&{}&{}\\
	{}&{}&{}\\
}}
\right) -
\left(
\parbox{6pt}{
\xymatrix@R=6pt@C=6pt{
	{}&{}\ar@{-}[ddr]&{}\\
	{}\ar@{-}[ur]&{}&{}\\
	{}&{}&{}\ar@{-}[ull]\\
}}
\right) -
\left(
\parbox{6pt}{
\xymatrix@R=6pt@C=6pt{
	{}\ar@{-}[drr]&{}&{}\\
	{}&{}&{}\ar@{-}[dl]\\
	{}&{}\ar@{-}[uul]&{}\\
}}
\right) 
\]
Точная формула%
\footnote{Для больших размерностей чем $3$ на $3$ явная формула не пригодна из-за слишком большого числа слагаемых.
Даже с вычислительной точки зрения.}
\[
\det A = a_{11}a_{22}a_{33} + a_{12}a_{23}a_{31} + a_{13}a_{21}a_{32} - 
a_{13}a_{22}a_{31} - a_{12}a_{21}a_{33} - a_{11}a_{23}a_{32}
\]
\end{enumerate}

\paragraph{Треугольные матрицы}

\begin{claim}\label{claim::DetUpperTr}
Для любых верхне и нижне треугольных матрицы верны следующие формулы:
\[
\det
\begin{pmatrix}
{\lambda_1}&{\ldots}&{*}\\
{}&{\ddots}&{\vdots}\\
{}&{}&{\lambda_n}\\
\end{pmatrix}
 = 
\lambda_1 \ldots \lambda_n
\quad
\det
\begin{pmatrix}
{\lambda_1}&{}&{}\\
{\vdots}&{\ddots}&{}\\
{*}&{\ldots}&{\lambda_n}\\
\end{pmatrix}
 = 
 \lambda_1 \ldots \lambda_n
\]
В частности $\det E = 1$.
\end{claim}
\begin{proof}
Я докажу утверждение для верхнетреугольных матриц, нижнетреугольный случай делается аналогично.
Для доказательства надо посчитать определитель по определению и увидеть, что только одно слагаемое соответствующее тождественной перестановке является не нулем.
Действительно, рассмотрим выражение $a_{1\sigma(1)}\ldots a_{n\sigma(n)}$.
Посмотрим когда это выражение не ноль.
Последний множитель $a_{n\sigma(n)}$ лежит в последней строке и должен быть не ноль.
Для этого должно выполняться $\sigma(n) = n$.
Теперь $a_{n-1\sigma(n-1)}$ должен быть не ноль.
Так как $\sigma(n) = n$, то $\sigma(n - 1)\neq n$.
А значит, чтобы $a_{n-1 \sigma(n-1)}$ был не ноль, остается только один случай $\sigma(n-1) = n-1$.
Продолжая аналогично, мы видим, что $\sigma(i) = i$ для всех строк $i$.
\end{proof}

\subsection{Свойства определителя}

\paragraph{Определитель и транспонирование}

Прежде чем перейти к доказательству следующего утверждения сделаем одно полезное наблюдение.
Если мы возьмем две произвольные перестановки $\sigma,\tau\in\Sym{n}$ и матрицу $A\in\Matrix{n}$, то выражения $a_{\tau(1) \sigma(\tau (1))}\ldots a_{\tau(n) \sigma(\tau(n))}$ совпадает с выражением $a_{1\sigma(1)}\ldots a_{n\sigma(n)}$ с точностью до перестановки сомножителей.
Это делается методом пристального взгляда: замечаем что каждый сомножитель одного выражения ровно один раз встречается в другом и наоборот.

\begin{claim}\label{claim::DetTranspose}
Пусть $A\in \Matrix{n}$, тогда $\det A = \det A^t$.
\end{claim}
\begin{proof}
Посчитаем по определению $\det A^t$, получим
\[
\det A^t = \sum_{\sigma\in\Sym{n}}\sgn(\sigma)a_{\sigma(1)1} \ldots a_{\sigma(n)n} = 
\sum_{\sigma\in\Sym{n}}\sgn(\sigma)a_{\sigma(1)\sigma^{-1}(\sigma(1))} \ldots a_{\sigma(n)\sigma^{-1}(\sigma(n))}
\]
Теперь применим наше замечание перед доказательством:
\[
a_{\sigma(1)\sigma^{-1}(\sigma(1))} \ldots a_{\sigma(n)\sigma^{-1}(\sigma(n))}
=
a_{1 \sigma^{-1}(1)}\ldots a_{n\sigma^{-1}(n)}
\]
Значит
\[
\det A^t = \sum_{\sigma\in\Sym{n}}\sgn(\sigma) a_{1 \sigma^{-1}(1)}\ldots a_{n\sigma^{-1}(n)}
\]
Вспомним, что $\sgn(\sigma) = \sgn(\sigma^{-1})$.
Следовательно:
\[
\det A^t = \sum_{\sigma\in\Sym{n}}\sgn(\sigma^{-1}) a_{1 \sigma^{-1}(1)}\ldots a_{n\sigma^{-1}(n)}
\]
Теперь, если $\sigma$ пробегает все перестановки, то $\sigma^{-1}$ тоже пробегает все перестановки, так как отображение $\Sym{n}\to \Sym{n}$ по правилу $\sigma\mapsto \sigma^{-1}$ является биекцией.%
\footnote{Оно биекция, так как имеет обратное -- оно само.}
То есть мы можем сделать замену $\tau = \sigma^{-1}$ и приходим к выражению
\[
\det A^t = \sum_{\tau\in\Sym{n}}\sgn(\tau)a_{1\tau(1)}\ldots a_{n\tau(n)}
\]
Последнее в точности совпадает с определением $\det A$.
\end{proof}

Отметим, что если мы доказали какое-то свойство определителя для столбцов, то это утверждение автоматически гарантирует, что такое же свойство выполнено и для строк.
И наоборот, если что-то сделано для строк, то это автоматом следует для столбцов.

\subsection{Полилинейность и кососимметричность определителя}

Сейчас мы докажем, что определитель обладает всеми свойствами (II) и (II').
В силу утверждения~\ref{claim::DetTranspose} нам достаточно показать только (II).

\begin{claim}\label{claim::DetPolyAnti}
Отображение $\det\colon \Matrix{n}\to \mathbb R$ рассматриваемое как отображение столбцов матрицы является полилинейным и кососимметричным, т.е. удовлетворяет следующим свойствам:
\begin{enumerate}
\item $\det(A_1,\ldots, A_i + A_i', \ldots, A_n) = \det(A_1,\ldots, A_i, \ldots, A_n) + \det(A_1,\ldots,A_i', \ldots, A_n)$ для любого $i$.

\item $\det(A_1,\ldots, \lambda A_i, \ldots, A_n) = \lambda \det(A_1,\ldots, A_i, \ldots, A_n)$ для любого $i$ и любого $\lambda\in\mathbb R$.

\item $\det(A_1,\ldots, A_i, \ldots, A_j, \ldots, A_n) = -\det(A_1,\ldots, A_j, \ldots, A_i, \ldots, A_n)$ для любых различных $i$ и $j$.

\item $\det E = 1$.
\end{enumerate}
\end{claim}
\begin{proof}
Мы знаем, что 
\[
\det A = \det A^t = \sum_{\sigma\in\Sym{n}}\sgn(\sigma)a_{\sigma(1)1} \ldots a_{\sigma(i)i}\ldots a_{\sigma(n)n}
\]
Проверим свойство (1):
\begin{gather*}
\det(A_1,\ldots, A_i + A_i', \ldots, A_n) = \sum_{\sigma\in\Sym{n}}\sgn(\sigma)a_{\sigma(1)1} \ldots \left(a_{\sigma(i)i} + a_{\sigma(i)i}'\right)\ldots a_{\sigma(n)n}=\\
= \sum_{\sigma\in\Sym{n}}\sgn(\sigma)a_{\sigma(1)1} \ldots a_{\sigma(i)i}\ldots a_{\sigma(n)n} +\sum_{\sigma\in\Sym{n}}\sgn(\sigma)a_{\sigma(1)1} \ldots a_{\sigma(i)i}'\ldots a_{\sigma(n)n} =\\
= \det(A_1,\ldots, A_i, \ldots, A_n) + \det(A_1,\ldots,A_i', \ldots, A_n)
\end{gather*}
Теперь свойство (2):
\[
\det(A_1,\ldots, \lambda A_i, \ldots, A_n) = \sum_{\sigma\in\Sym{n}}\sgn(\sigma)a_{\sigma(1)1} \ldots \left(\lambda a_{\sigma(i)i}\right)\ldots a_{\sigma(n)n}= \lambda \det(A_1,\ldots,  A_i, \ldots, A_n) 
\]
Для проверки свойства (3) введем следующее обозначение.
Пусть $\tau\in\Sym{n}$ обозначает транспозицию $(i,j)$.
Тогда посчитаем определитель с переставленными столбцами $i$ и $j$ местами:
\begin{gather*}
\det (A_1,\ldots, A_j, \ldots, A_i, \ldots, A_n) = \sum_{\sigma\in\Sym{n}}\sgn(\sigma) a_{\sigma(1)1}\ldots a_{\sigma(i)j}\ldots a_{\sigma(j)i}\ldots a_{\sigma(n)n} = \\
=  \sum_{\sigma\in\Sym{n}}\sgn(\sigma) a_{\sigma(1)\tau(1)}\ldots a_{\sigma(i)\tau(i)}\ldots a_{\sigma(j)\tau(j)}\ldots a_{\sigma(n)\tau(n)} = \\
= \sum_{\sigma\in\Sym{n}} \sgn(\sigma) a_{\sigma(\tau^{-1}(1))1} \ldots a_{\sigma(\tau^{-1}(n))n} = -\sum_{\sigma\in\Sym{n}} \sgn(\sigma\tau^{-1}) a_{\sigma(\tau^{-1}(1))1} \ldots a_{\sigma(\tau^{-1}(n))n}
\end{gather*}
Здесь при переходе от второй строчки к третьей мы воспользовались замечанием перед упражнением~\ref{claim::DetTranspose}.
Так как отображение $\Sym{n}\to \Sym{n}$ по правилу $\sigma\mapsto \sigma \tau^{-1}$ является биекцией, то если $\sigma$ пробегает все перестановки, то и $\sigma\tau^{-1}$ пробегает все перестановки.
А значит, делая замену $\rho = \sigma \tau^{-1}$, получаем
\[
-\sum_{\sigma\in\Sym{n}} \sgn(\sigma\tau^{-1}) a_{\sigma(\tau^{-1}(1))1} \ldots a_{\sigma(\tau^{-1}(n))n} = -\det (A_1,\ldots, A_i, \ldots, A_j,\ldots A_n)
\]

(4) Это непосредственно следует из определения, либо если хотите, то можно сослаться на утверждение~\ref{claim::DetUpperTr}.
\end{proof}

\begin{claim*}
Если $A\in\Matrix{n}$ имеет нулевой столбец или нулевую строку, то $\det A = 0$.
\end{claim*}
\begin{proof}
Пусть $A$ имеет нулевой столбец.
Мы знаем, что $\det$ -- полилинейная функция.
Значит, если мы умножим нулевой столбец на $-1$, определитель должен поменять знак.
С другой стороны, если мы умножим нулевой столбец на любое число, он не поменяется и определитель не должен поменяться.
Значит по безысходности определитель должен быть $0$.
\end{proof}


\paragraph{Определитель от элементарных матриц}

\begin{claim}
Верны следующие утверждения:
\begin{enumerate}
\item $\det (S_{ij}(\lambda)) = 1$, где $S_{ij}(\lambda)\in\Matrix{n}$ -- матрица элементарного преобразования первого типа.

\item $\det (U_{ij}) = -1$, где $U_{ij}\in\Matrix{n}$ -- матрица элементарного преобразования второго типа.

\item $\det(D_i(\lambda)) = \lambda$, где $D_{i}(\lambda)\in\Matrix{n}$ -- матрица элементарного преобразования третьего типа.
\end{enumerate}
\end{claim}
\begin{proof}
(1) Является следствием для случая верхне- и нижнетреугольных матриц.

(2) Так как $U_{ij}$ получается из единичной матрицы перестановкой $i$-го и $j$-го столбцов, то результат следует из кососимметричности определителя.

(3) Следует из полилинейности определителя -- свойство (II)~(2).
\end{proof}




\subsection{Полилинейные кососимметрические отображения}

Все утверждения в этом разделе доказываются для строк.
Соответствующие утверждения для столбцов доказываются аналогично.
Их формулировки и доказательства я оставляю в качестве упражнения.

\begin{claim}\label{claim::PolyAntiAndElementary}
Пусть $\phi\colon \Matrix{n}\to \mathbb R$ -- полилинейное кососимметрическое отображение по строкам матриц, т.е. удовлетворяет следующим свойствам:%
\footnote{Здесь через $A_i$ обозначаются строки матрицы $A$ идущие сверху вниз.}
\begin{enumerate}
\item $\phi(A_1,\ldots, A_i + A_i', \ldots, A_n) = \phi(A_1,\ldots, A_i, \ldots, A_n) + \phi(A_1,\ldots,A_i', \ldots, A_n)$ для любого $i$.

\item $\phi(A_1,\ldots, \lambda A_i, \ldots, A_n) = \lambda \phi(A_1,\ldots, A_i, \ldots, A_n)$ для любого $i$ и любого $\lambda\in\mathbb R$.

\item $\phi(A_1,\ldots, A_i, \ldots, A_j, \ldots, A_n) = -\phi(A_1,\ldots, A_j, \ldots, A_i, \ldots, A_n)$ для любых различных $i$ и $j$.
\end{enumerate}
Тогда $\phi(UA) = \det(U)\phi(A)$ для любой матрицы $A\in\Matrix{n}$ и любой элементарной матрицы $U\in\Matrix{n}$.
\end{claim}
\begin{proof}
Случай $U = S_{ij}(\lambda)$.
\begin{gather*}
\phi(S_{ij}(\lambda)A) = \phi(A_1,\ldots, A_i + \lambda A_j, \ldots, A_j, \ldots, A_n) = \\
= \phi(A_1,\ldots, A_i, \ldots, A_j, \ldots, A_n) + \lambda\phi(A_1,\ldots, A_j, \ldots, A_j, \ldots, A_n) =  \phi(A) = \det(S_{ij}(\lambda))\phi(A)
\end{gather*}

Случай $U= U_{ij}$.
\begin{gather*}
\phi(U_{ij}A) = \phi(A_1,\ldots, A_j, \ldots, A_i, \ldots, A_n) = -\phi(A_1,\ldots, A_i, \ldots, A_j, \ldots, A_n) = -\phi(A) = \det(U_{ij})\phi(A)
\end{gather*}

Случай $U=D_i(\lambda)$.
\begin{gather*}
\phi(D_i(\lambda)A) = \phi(A_1,\ldots,\lambda A_i,\ldots, A_n) = \lambda \phi(A_1,\ldots, A_i,\ldots, A_n) = \lambda \phi(A) = \det(D_i(\lambda))\phi(A)
\end{gather*}
\end{proof}

\paragraph{Определитель и элементарные матрицы}

Заметим, что по утверждению~\ref{claim::DetPolyAnti}, определитель тоже является полилинейной и кососимметрической функцией.
Потому доказанное утверждение в частности означает, что $\det(UA) = \det(U)\det(A)$ для любой матрицы $A\in\Matrix{n}$ и любой элементарной матрицы $U\in\Matrix{n}$.

\paragraph{Подсчет определителя}

Предыдущее замечание позволяет дать эффективный способ вычисления определителя методом Гаусса.
Мы берем матрицу $A$ и приводим ее к ступенчатому виду попутно запоминая как изменился определитель по сравнению с определителем изначальной матрицы.
Если же мы будем использовать только элементарные преобразования первого типа, то определитель вовсе меняться не будет.
Ступенчатый вид матрицы всегда верхне треугольный.
Там определитель считается как произведение диагональных элементов.


\paragraph{Следствия утверждения~\ref{claim::PolyAntiAndElementary}}

\begin{claim}[Единственность для полилинейных кососимметричных]\label{claim::PolyAntiUnique}
Пусть $\phi\colon \Matrix{n}\to \mathbb R$ -- полилинейное кососимметрическое отображение по строкам матриц.
Тогда $\phi(X) = \det(X)\phi(E)$.
В частности, если $\phi(E) = 1$, то $\phi = \det$.
\end{claim}
\begin{proof}
Пусть $X\in\Matrix{n}$ -- произвольная матрица, тогда ее можно элементарными преобразованиями строк привести к улучшенному ступенчатому виду.
Последнее означает, что $X = U_1 \ldots U_k S$, где $S$ -- матрица улучшенного ступенчатого вида, а $U_i$ -- матрицы элементарных преобразований.
Применим к этому равенству отдельно $\phi$ и отдельно $\det$, получим
\begin{align*}
\phi(X) &= \det(U_1)\ldots \det(U_k)\phi(S)\\
\det(X) &= \det(U_1)\ldots \det(U_k)\det(S)
\end{align*}
Теперь для матрицы $S$ у нас есть два варианта: либо $S$ единичная, либо содержит нулевую строку.

Пусть $S = E$, тогда
\begin{align*}
\phi(X) &= \det(U_1)\ldots \det(U_k)\phi(E)\\
\det(X) &= \det(U_1)\ldots \det(U_k)
\end{align*}
Откуда и получаем требуемое $\phi(X) = \det(X)\phi(E)$.

Пусть теперь $S$ имеет нулевую строку.
Тогда из полилинейности определителя и $\phi$ следует, что $\phi(S) = 0 =\det(S)$.%
\footnote{При умножении на $-1$ нулевой строки с одной стороны функция должна поменять знак, а с другой не измениться.}
Что тоже влечет равенство $\phi(X) = \det(X)\phi(E)$.
\end{proof}

